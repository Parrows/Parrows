\begin{abstract}
Arrows were introduced in John Hughes paper as an alternative to monads for API design \citHughes. In the paper Hughes describes that arrows have one powerful additional property when compared to monads for API design: Extensibility. In this paper we will show how this property can be used to add parallelism capabilities to different arrow-types.
\\\\
First, we give an introduction to some of the possible ways to add parallelism to Haskell programs. Then, we give the basic definition of Arrows, which is followed up by the introduction of some utility functions used in this paper. Next, we introduce the \code{ArrowParallel} typeclass together with backends for it written with the parallel Haskells introduced earlier, finishing up with the benefits of this new way of writing parallel programs. After this we give the definition of several parallel skeletons. Then, we introduce some syntactic sugar to mimic the sequential arrow combinators introduced by Hughes \citHughes ~but with parallelism added. We also give benchmarks of our newly created parallel Haskell based on arrows. Finally we give a short conclusion of what we managed to achieve.
\end{abstract}