\section{Conclusion}
\label{sec:conclusion}
Arrows are a generic concept that allows for powerful composition combinators. To our knowledge we are the first ones to represent parallel computation with arrows.\olcomment{that strange arrows-based robot interaction paper from 1993 or so! clearly discuss in related work}

Arrows turn out to be a useful tool for composing in parallel programs. We do not have to introduce new monadic types that wrap the computation. Instead use arrows just like regular sequential pure functions. 
%
This work features multiple parallel backends: the already available parallel Haskell flavours. Parallel Arrows feature an implementation of the \code{ArrowParallel} interface for Multicore Haskell, \code{Par} monad, and Eden. With our approach parallel programs can be ported across these flavours with no effort.
%
%
Performancewise, Parallel Arrows are on par with existing parallel Haskells, as they do not introduce any notable overhead.\olcomment{PROOFS. Many proofs in benchmarks!}

\mbcomment{ArrowApply (or equivalent) are needed because we basically want to be able to produce intermediary results, this is by definition of the parallel evaluation combinators}

\olcomment{Remove websites from citations, put them into footnotes!}

\olcomment{Parrows + accelerate = love? Metion port to Frege. Mention the Par monad troubles.}