\section{Utility Functions}\label{utilfns}
Before we go into detail on parallel arrows, we introduce some utility combinators first, that will help us later: map and zipWith on arrows.
\\
The map combinator lifts any arrow \code{arr a b} to an arrow \code{arr [a] [b]},
\begin{lstlisting}[frame=htrbl]
-- from http://www.cse.chalmers.se/~rjmh/afp-arrows.pdf
mapArr :: ArrowChoice arr => arr a b -> arr [a] [b]
mapArr f =
	arr listcase >>>
	arr (const []) ||| (f *** mapArr f >>> arr (uncurry (:)))
	where
		listcase [] = Left ()
		listcase (x:xs) = Right (x,xs)
\end{lstlisting}
and zipWith lift any arrow \code{arr (a, b) c} to an arrow \code{arr ([a], [b]) [c]}.
\begin{lstlisting}[frame=htrbl]
zipWithArr :: ArrowChoice arr => arr (a, b) c -> arr ([a], [b]) [c]
zipWithArr f = (arr $ \(as, bs) -> zipWith (,) as bs) >>> mapArr f
\end{lstlisting}
These two combinators make use of the \code{ArrowChoice} typeclass, which allows us to use the \code{|||} combinator. It takes two arrows \code{arr a c} and \code{arr b c} and combines them into a new arrow \code{arr (Either a b) c} which pipes all \code{Left a}'s to the first arrow and all \code{Right b}'s to the second arrow.
\begin{lstlisting}[frame=htrbl]
(|||) :: ArrowChoice arr a c -> arr b c -> arr (Either a b) c
\end{lstlisting}
With the zipWithArr combinator we can also write a combinator \code{listApp}, which lifts a list of arrows \code{[arr a b]} to an arrow \code{arr [a] [b]}.
\begin{lstlisting}[frame=htrbl]
listApp :: (ArrowChoice arr, ArrowApply arr) => [arr a b] -> arr [a] [b]
listApp fs = (arr $ \as -> (fs, as)) >>> zipWithArr app
\end{lstlisting}
This combinator also makes use of the \code{ArrowApply} typeclass which allows us to evaluate arrows with \code{app :: arr (arr a b, a) c}.
