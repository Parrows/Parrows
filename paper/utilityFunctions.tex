\section{Utility Functions}\label{utilfns}
To be able to go into detail on parallel arrows, we introduce some utility combinators first, that will help us later: \lstinline{map}, \lstinline{foldl} and \lstinline{zipWith} on arrows.

The \code{mapArr} combinator lifts any arrow \code{arr a b} to an arrow \code{arr [a] [b]} \cite{programming_with_arrows}:
\begin{lstlisting}[frame=htrbl]
mapArr :: ArrowChoice arr => arr a b -> arr [a] [b]
mapArr f =
	arr listcase >>>
	arr (const []) ||| (f *** mapArr f >>> arr (uncurry (:)))
	where listcase [] = Left ()
	      listcase (x:xs) = Right (x,xs)
\end{lstlisting}
Similarly, we can also define \code{foldlArr} that lifts any arrow \code{arr (b, a) b} with a neutral element \code{b} to \code{arr [a] b}:
\begin{lstlisting}[frame=htrbl]
foldlArr :: (ArrowChoice arr, ArrowApply arr) => arr (b, a) b -> b -> arr [a] b
foldlArr f b =
	arr listcase >>>
	arr (const b) |||
		(first (arr (\a -> (b, a)) >>> f >>> arr (foldlArr f)) >>> app)
	where listcase [] = Left []
	      listcase (x:xs) = Right (x,xs)
\end{lstlisting}
Finally, with the help of \code{mapArr}, we can define \lstinline{zipWithArr} that lifts any arrow \code{arr (a, b) c} to an arrow \code{arr ([a], [b]) [c]}.
\begin{lstlisting}[frame=htrbl]
zipWithArr :: ArrowChoice arr => arr (a, b) c -> arr ([a], [b]) [c]
zipWithArr f = (arr $ \(as, bs) -> zipWith (,) as bs) >>> mapArr f
\end{lstlisting} %$ %% formatting
These combinators make use of the \code{ArrowChoice} type class, it provides the \code{|||} combinator. The latter takes two arrows \code{arr a c} and \code{arr b c} and combines them into a new arrow \code{arr (Either a b) c} which pipes all \code{Left a}'s to the first arrow and all \code{Right b}'s to the second arrow.
\begin{lstlisting}[frame=htrbl]
(|||) :: ArrowChoice arr a c -> arr b c -> arr (Either a b) c
\end{lstlisting}
With the zipWithArr combinator we can also write a combinator \code{listApp}, which lifts a list of arrows \code{[arr a b]} to an arrow \code{arr [a] [b]},
\begin{lstlisting}[frame=htrbl]
listApp :: (ArrowChoice arr, ArrowApply arr) => [arr a b] -> arr [a] [b]
listApp fs = (arr $ \as -> (fs, as)) >>> zipWithArr app
\end{lstlisting}% $ %% formatting
which also makes use of the \code{ArrowApply} typeclass that allows us to evaluate arrows with \code{app :: arr (arr a b, a) c}.

The definition of \code{unshuffle} is
\begin{lstlisting}[frame=htrbl]
unshuffle :: (Arrow arr) => Int -> arr [a] [[a]]
unshuffle n = arr (\xs -> [takeEach n (drop i xs) | i <- [0..n-1]])

takeEach :: Int -> [a] -> [a]
takeEach n [] = []
takeEach n (x:xs) = x : takeEach n (drop (n-1) xs)
\end{lstlisting}
while \code{shuffle} is defined as:
\begin{lstlisting}[frame=htrbl]
shuffle :: (Arrow arr) => arr [[a]] [a]
shuffle = arr (concat . transpose)
\end{lstlisting}
These functions were taken from Eden skeleton source code \cite{eden_skel_shuffle} and lifted to Arrows.


The helper functions for \code{torus} are
\begin{lstlisting}[frame=htrbl]
uncurry3 :: (a -> b -> c -> d) -> (a, (b, c)) -> d
uncurry3 f (a, (b, c)) = f a b c

lazyzip3 :: [a] -> [b] -> [c] -> [(a, b, c)]
lazyzip3 as bs cs = zip3 as (lazy bs) (lazy cs)

threetotwo :: (Arrow arr) => arr (a, b, c) (a, (b, c))
threetotwo = arr $ \ ~(a, b, c) -> (a, (b, c))
\end{lstlisting} % $ %% formatting


%%% Local Variables:
%%% mode: latex
%%% TeX-master: "main"
%%% End:
