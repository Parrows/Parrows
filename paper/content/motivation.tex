%\section{Motivation}
%Arrows were introduced in John Hughes paper as a general interface for computation and therefore as an alternative to monads for API design \citHughes. In the paper Hughes describes how arrows are a generalization of monads and how they are not as restrictive. In this paper we will use this concept to express parallelism.

\section{Introduction}
\label{sec:introduction}
\olcomment{todo, reuse 5.5, and more}

blablabla arrows, parallel, haskell.

\paragraph{Contribution}

\olcomment{HIT HERE REALLY STRONG}

%\subsection{Impact of parallel Arrows}
%\olcomment{move this to Contributions in the front or something}
\mbcomment{different, how?}
We wrap parallel Haskells inside of our |ArrowParallel| interface, but why do we aim to abstract parallelism this way and what does this approach do better than the other parallel Haskells?
\begin{itemize}
	\item \textbf{Arrow DSL benefits}:
	With the |ArrowParallel| typeclass we do not lose any benefits of using arrows as |parEvalN| is a yet another Arrow combinator. The resulting Arrow can be used in the same way a potential serial version could be used. This is a big advantage of this approach, especially compared to the monad solutions as we do not introduce any new types. We can just \enquote{plug} in parallel parts into sequential Arrow-based programs without having to change anything.
	\item \textbf{Abstraction}:
	With the |ArrowParallel| typeclass, we abstract all parallel implementation logic away from the business logic. This leaves us in the beautiful situation of being able to write our code against the interface of the typeclass without being bound to any parallel Haskell. So as an example, during development, we can run the program in a simple GHC-compiled variant and afterwards deploy it on a cluster by converting it into an Eden version, by just replacing the actual |ArrowParallel| instance.
\end{itemize}


\paragraph{Structure}
The remaining text is structures as follows. Section~\ref{sec:background} briefly introduces known parallel Haskell flavours (Sec.~\ref{sec:parEvalNIntro}) and gives an overview of Arrows to the reader (Sec.~\ref{sec:arrows}). Section~\ref{sec:related-work} discusses related work. Section~\ref{sec:parallel-arrows} defines Parallel Arrows and presents a basic interface. Section~\ref{futures} defines Futures for Parallel Arrows, this concept enables better communication. Section~\ref{sec:map-skeletons} presents some basic algorithmic skeletons  in our newly defined dialect: parallel |map| with and without load balancing. More advanced skeletons are showcased in Section~\ref{sec:topology-skeletons} (|pipe|, |ring|, |torus|). Section~\ref{sec:benchmarks} shows the benchmark results. Section~\ref{sec:conclusion} discusses future work and concludes.

%%% Local Variables:
%%% mode: latex
%%% TeX-master: "main"
%%% End:
