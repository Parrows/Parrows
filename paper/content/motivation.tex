%\section{Motivation}
%Arrows were introduced in John Hughes paper as a general interface for computation and therefore as an alternative to Monads for API design \citHughes. In the paper Hughes describes how Arrows are a generalization of Monads and how they are not as restrictive. In this paper we will use this concept to express parallelism.

\section{Introduction}
\label{sec:introduction}

Parallel functional languages have a long history of being used for experimenting with novel parallel programming paradigms. Haskell, which we focus on in this paper, has  several mature implementations. We regard here in-depth
Glasgow parallel Haskell or short GpH (its Multicore SMP implementation, in particular), the
|Par| Monad, and Eden, a distributed memory parallel Haskell. These
languages represent orthogonal approaches. Some use a monad, even if
only for the internal representation. Some introduce additional
language constructs. Section \ref{sec:parallelHaskells} gives a short overview over these languages.

A key novelty in this paper is to use Arrows to represent parallel computations. They seem a natural fit as they can be thought of as a more general function arrow (|->|) and serve as general interface to computations while not being as restrictive as Monads \citep{HughesArrows}. Section \ref{sec:arrows} gives a short introduction to Arrows.

We provide an Arrows-based type class and implementations for the three above mentioned parallel Haskells.
Instead of 
introducing a new low-level parallel backend to implement our
Arrows-based interface, we define a shallow-embedded DSL for Arrows. This DSL
is defined as a common interface with varying implementations in
the existing parallel Haskells. Thus, we not only define a parallel programming interface in a
novel manner -- we tame the zoo of parallel Haskells. We provide a
common, very low-penalty programming interface that allows to switch
the parallel backends at will. The induced penalty was in the single-digit percent range, with means over the varying cores configuration typically under 1\% overhead in our measurements (Section~\ref{sec:benchmarks}). Further backends based on HdpH or a Frege implementation (on the Java Virtual Machine) are viable, too.

\paragraph{Contributions}
%
%\olcomment{HIT HERE REALLY STRONG}
%
%\subsection{Impact of parallel Arrows}
%\olcomment{move this to Contributions in the front or something}
We propose an Arrow-based encoding for parallelism based on a new Arrow combinator |parEvalN :: [arr a b] -> arr [a] [b]|. A parallel Arrow is still an Arrow, hence the resulting parallel Arrow can still be used in the same way as a potential sequential version. In this paper we evaluate the expressive power of such a formalism in the context of parallel programming.

\begin{itemize}
\item We introduce a parallel evaluation formalism using Arrows. One big advantage of our specific approach is that we do not have to introduce any new types, facilitating composability (Section~\ref{sec:parallel-arrows}).
\item We show that PArrow programs can readily exploit multiple parallel language implementations. We demonstrate the use of GpH, a |Par| Monad, and Eden. We do not re-implement all the parallel internals, as we host this functionality in the |ArrowParallel| type class, which abstracts all parallel implementation logic. The backends can easily be swapped, so we are not bound to any specific one.

This has many practical advantages. For example, during development we can run the program in a simple GHC-compiled variant using a GpH backend and afterwards deploy it on a cluster by converting it into an Eden program, by just replacing the |ArrowParallel| instance and compiling with Eden's GHC variant. (Section~\ref{sec:parallel-arrows})
\item We extend the PArrows formalism with |Future|s to enable direct communication of data between nodes in a distributed memory setting similar to Eden's Remote Data (|RD|) \citet{Dieterle2010}. Direct communication is useful in a distributed memory setting because it allows for inter-node communication without blocking the master-node. (Section~\ref{sec:futures})
\item We demonstrate the expressiveness of PArrows by using them to define common algorithmic skeletons ((Section~\ref{sec:skeletons}), and by using these skeletons to implement four benchmarks (Section~\ref{sec:benchmarks}).
\item We practically demonstrate that Arrow parallelism has a low performance overhead compared with existing approaches, \eg with mean of mean overhead of less than $3.5\%, 0.8\%$ for all benchmarks with GpH and Eden respectively. Compared to |Par| Monad we virtually noticed no overhead with a mean of mean overheads of less than $0\%$ in all benchmarks. (Section~\ref{sec:benchmarks}).
\end{itemize}

PArrows are open source and available from \url{https://github.com/s4ke/Parrows}.