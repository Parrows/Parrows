\usepackage{tikz}

\usepackage{filecontents}
\usepackage{pgfplots, pgfplotstable}
\usepgfplotslibrary{statistics}

\usepackage{makecell}

\usepackage[T1]{fontenc}
\usepackage[utf8]{inputenc}

\usepackage{csvsimple}

\usepackage{tablefootnote}

%\usepackage{color}
\usepackage{hyperref}

%\usepackage{subcaption}

\usepackage[round,sectionbib]{natbib}
%% https://tex.stackexchange.com/questions/148992/changing-the-heading-style-of-references-section
%% and jfp1.cls
\renewcommand\refname{References}
\makeatletter
\renewcommand\bibsection{%
  \section*{\refname}%
}%
\makeatother

\usepackage{csquotes}

% \usepackage{placeins} %% unneeded

\usepackage{microtype}
\DisableLigatures[>,<]{encoding = T1,family=tt*} 

% für Listings
%\usepackage{listings}


\renewcommand{\cite}[1]{\citep{#1}}

\newcommand{\citHughes}{\citep{HughesArrows}}

%\newcommand{\inlinecode}[1]{\texttt{#1}}
\newcommand{\inlinecode}[1]{\emph{#1}} %%% booo
%%% |code| is inline code in lhs2tex

%\newcommand{\fixme}[1]{\colorbox{red}{#1}}

%%%%%% more fun and typography

\usepackage{xifthen}

\newboolean{anonymous}
\setboolean{anonymous}{False}

\usepackage[british]{babel}

\usepackage{siunitx}

\usepackage{array}
\newcolumntype{L}[1]{>{\raggedright\let\newline\\\arraybackslash\hspace{0pt}}m{#1}}
\newcolumntype{C}[1]{>{\centering\let\newline\\\arraybackslash\hspace{0pt}}m{#1}}
\newcolumntype{R}[1]{>{\raggedleft\let\newline\\\arraybackslash\hspace{0pt}}m{#1}}

%%\usepackage{tipa} %% causes problems with lhs2tex?!

\usepackage{xspace}
\usepackage{xcolor}
\newcommand{\hl}[1]{\textcolor{red}{#1}}
%\newcommand{\comm}[2]{\textcolor{red}{\bfseries #1: #2}}
\newcommand{\comm}[2]{}
\newcommand{\olcomment}[1]{\comm{OL}{#1}}
\newcommand{\mbcomment}[1]{\comm{MB}{#1}}
\newcommand{\ptcomment}[1]{\comm{Phil}{#1}}
%\newcommand{\done}{\xspace\hl{done!}\xspace}
\newcommand{\fixme}{\mbcomment}

%\renewcommand{\olcomment}[1]{}
%\renewcommand{\mbcomment}[1]{}

\DeclareRobustCommand{\xth}{\textsuperscript{th}\xspace}
\DeclareRobustCommand{\st}{\textsuperscript{st}\xspace}
\DeclareRobustCommand{\nd}{\textsuperscript{nd}\xspace}
\DeclareRobustCommand{\xrd}{\textsuperscript{rd}\xspace}

%% kuerzel
%\newcommand{\todo}{\textcolor{red}{\bfseries TODO!}\xspace}
\newcommand{\done}{\textcolor{green}{\bfseries done!}\xspace}
\DeclareRobustCommand{\hairspn}{\hspace{1pt}\nolinebreak}% hair space with no break

\newcommand{\seconds}{\ensuremath{s}}

% \DeclareRobustCommand{\ie}{{i.\hairspn{}e.\nopagebreak[4] }}
% \DeclareRobustCommand{\eg}{{e.\hairspn{}g.\nopagebreak[4] }}
% \DeclareRobustCommand{\fe}{{f.\hairspn{}e.,}\nopagebreak[4] }
\DeclareRobustCommand{\ie}{{i.\hairspn{}e.~}}
\DeclareRobustCommand{\eg}{{e.\hairspn{}g.~}}
\DeclareRobustCommand{\fe}{{f.\hairspn{}e.,~}}
\DeclareRobustCommand{\ad}{{A.\hairspn{}D.\xspace}}
\DeclareRobustCommand{\bcc}{{A.\hairspn{}C.\xspace}}
\DeclareRobustCommand{\wrt}{w.\hairspn{}r.\hairspn{}t.~}
\DeclareRobustCommand{\etc}{etc.\ }
\DeclareRobustCommand{\cf}{\textit{cf.~}}
\DeclareRobustCommand{\viz}{\textit{viz.~}}
%\DeclareRobustCommand{\hof}{higher-or\-der function\xspace}

% %% the numberings
\DeclareRobustCommand{\xth}{\textsuperscript{th}\xspace}
\DeclareRobustCommand{\st}{\textsuperscript{st}\xspace}
\DeclareRobustCommand{\nd}{\textsuperscript{nd}\xspace}
\DeclareRobustCommand{\xrd}{\textsuperscript{rd}\xspace}

\newcommand{\tabsepbakup}{\tabcolsep}
%% for narrower tables

%% nice tables
\usepackage{booktabs}
\usepackage{multirow}

%% more microtype
\microtypecontext{spacing=nonfrench} %% log said so

\usepackage{subfig}


%% fight for pipepipepipe
\newcommand{\pipepipepipe}{\ensuremath{\mathbin{\mid\!\mid\!\mid}}\xspace}
%%% ^^^^ keep consistent with definition at the top of main.lhs
%%%\newcommand{\pipepipepipe}{\inlinecode{|||}\xspace}

%%% JFP requirements:
%%% Harvard citing style, "(Curry 1933)".
%%% code: identifies italic, keywords bold
%%% figures: eps(!) or ps (can convert pdf), eventually provide a greyscale version, color NOT in cmyk
