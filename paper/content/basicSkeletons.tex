\subsection{Extending the Interface}
\label{sec:extending-interface}
With the |ArrowParallel| type class in place and implemented, we can now define some further basic parallel interface functions. These are algorithmic skeletons that, however, mostly serve as a foundation to further, more specific algorithmic skeletons.

\subsubsection{Lazy |parEvalN|}
\begin{figure}[tb]
	\includegraphics[scale=0.7]{images/parEvalNLazy}
	\caption{Schematic depiction of |parEvalNLazy|.}
	\label{fig:parEvalNLazyImg}
\end{figure}
|parEvalN| fully traverses the list of passed Arrows as well as their inputs. Sometimes this might not be feasible, as it will not work on infinite lists of functions like e.g. |map (arr . (+)) [1..]| or just because we need the Arrows evaluated in chunks. |parEvalNLazy| (Figs.~\ref{fig:parEvalNLazyImg},~\ref{fig:parEvalNLazy}) fixes this. It works by first chunking the input from |[a]| to |[[a]]| with the given |chunkSize| in |arr (chunksOf chunkSize)|. These chunks are then fed into a list |[arr [a] [b]]| of chunk-wise parallel Arrows with the help of our lazy and sequential |evalN|. The resulting |[[b]]| is lastly converted into |[b]| with |arr concat|.
\begin{figure}[t]
\begin{code}
parEvalNLazy :: (ArrowParallel arr a b conf, ArrowChoice arr, ArrowApply arr) =>
	conf -> ChunkSize -> [arr a b] -> (arr [a] [b])
parEvalNLazy conf chunkSize fs =
	arr (chunksOf chunkSize) >>>
    evalN fchunks >>>
    arr concat
    where
      fchunks = map (parEvalN conf) (chunksOf chunkSize fs)
\end{code} %$ %% formatting
\caption{Definition of |parEvalNLazy|.}
\label{fig:parEvalNLazy}
\end{figure}

\subsubsection{Heterogeneous tasks}
\begin{figure}[tb]
	\includegraphics[scale=0.7]{images/parEval2}
	\caption{Schematic depiction of |parEval2|.}
	\label{fig:parEval2Img}
\end{figure}
We have only talked about the parallelization of Arrows of the same set of input and output types until now. But sometimes we want to parallelize heterogeneous types as well. We can implement such a |parEval2| combinator (Figs.~\ref{fig:parEval2Img},~\ref{fig:parEval2}) which combines two Arrows |arr a b| and |arr c d| into a new parallel Arrow |arr (a, c) (b, d)| quite easily with the help of the |ArrowChoice| type class. Here, the general idea is to use the |+++| combinator which combines two Arrows |arr a b| and |arr c d| and transforms them into |arr (Either a c) (Either b d)| to get a common Arrow type that we can then feed into |parEvalN|.

%%% Local Variables:
%%% mode: latex
%%% TeX-master: "main"
%%% End:
