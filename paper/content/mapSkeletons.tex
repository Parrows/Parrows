\section{Skeletons}
\label{sec:skeletons}
Now we have developed Parallel Arrows far enough to define some useful algorithmic skeletons that abstract typical parallel computations. While there are many possible skeletons to implement, we demonstrate the expressive power of PArrows here using four |map|-based and three toplogical skeletons.
%%% \FloatBarrier
\subsection{|map|-based Skeletons}
\label{sec:map-skeletons}
The essential differences between the mapping skeletons presented here are in terms of order of evaluation and work distribution but still provide the same semantics as a sequential |map|.

\paragraph{Parallel |map| and laziness.}
The |parMap| skeleton (Figs.~\ref{fig:parMapImg},~\ref{fig:parMap}) is probably the most common skeleton for parallel programs. We can implement it with |ArrowParallel| by repeating an Arrow |arr a b| and then passing it into |parEvalN| to obtain an Arrow |arr [a] [b]|.
Just like |parEvalN|, |parMap| traverses all input Arrows as well as the inputs.
Because of this, it has the same restrictions as |parEvalN| as compared to |parEvalNLazy|. So it makes sense to also have a |parMapStream| (Figs.~\ref{fig:parMapStreamImg},~\ref{fig:parMapStream}) which behaves like |parMap|, but uses |parEvalNLazy| instead of |parEvalN|. Implementing these skeletons is straightforward as in Appendix \ref{app:omitted} in Figs.\ref{fig:parMap} and \ref{fig:parMapStream}.

\begin{figure}[thb]
%farm
\includegraphics[scale=0.7]{images/farm}
\caption{|farm| depiction.}
\label{fig:farmImg}

\begin{code}
farm :: (ArrowParallel arr a b conf,
	ArrowParallel arr [a] [b] conf, ArrowChoice arr) =>
	conf -> NumCores -> arr a b -> arr [a] [b]
farm conf numCores f =
	unshuffle numCores >>>
	parEvalN conf (repeat (mapArr f)) >>>
	shuffle
\end{code}
\caption{|farm| definition.}
\label{fig:farm}

%farmChunk
\includegraphics[scale=0.7]{images/farmChunk}
\caption{|farmChunk| depiction.}
\label{fig:farmChunkImg}
\end{figure}

\paragraph{Statically load-balancing parallel |map|.}
Our |parMap| spawns every single computation in a new thread (at least for the instances of |ArrowParallel| we presented in this paper). This can be quite wasteful and a statically load-balancing |farm| (Figs.~\ref{fig:farmImg},~\ref{fig:farm}) that equally distributes the workload over |numCores| workers seems useful.
The definitions of the helper functions |unshuffle|, |takeEach|, |shuffle| (Fig.~\ref{fig:edenshuffleetc}) originate from an Eden skeleton\footnote{Available on Hackage under \url{https://hackage.haskell.org/package/edenskel-2.1.0.0/docs/src/Control-Parallel-Eden-Map.html}.}.

%\paragraph{Lazy statically load-balancing parallel map}
Since a |farm|  is basically just |parMap| with a different work distribution, it has the same restrictions as |parEvalN| and |parMap|. We can, however, define |farmChunk| (Figs.~\ref{fig:farmChunkImg},~\ref{fig:farmChunk}) which uses |parEvalNLazy| instead of |parEvalN|. It is basically the same definition as for |farm|, but with |parEvalNLazy| instead of |parEvalN|.

%%% Local Variables:
%%% mode: latex
%%% TeX-master: "main"
%%% End:
