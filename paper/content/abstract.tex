\begin{abstract}
Arrows are a general interface for computation and therefore form an alternative to monads for API design. We express parallelism using this concept in a novel way: We define an arrows-based language for parallelism and implement it using multiple parallel Haskells.
In this manner we are able to bridge across various parallel Haskells.

Additionally, our way of writing parallel programs has the benefit of being portable across flavours of parallel Haskells.
Furthermore, as each parallel computation is an arrow, which means that they can be composed and transformed as such.
We introduce some syntactic sugar to provide parallelism-aware arrow combinators.

To show that our arrow-based language is on par with the existing parallel languages, we also define several parallel skeletons with our framework. 
Benchmarks show that our framework does not induce too much overhead
performance-wise.
\olcomment{Summarize conclusions}
\end{abstract}
