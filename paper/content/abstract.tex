\begin{abstract}
Arrows are a general interface for computation and an alternative to Monads for API design. In contrast to Monad based parallelism, we explore the use of Arrows for specifying generalised parallelism. Specifically, we define an Arrow-based language and implement it using multiple parallel Haskells.

As each parallel computation is an Arrow, such parallel Arrows (PArrows) can be readily composed and transformed as such.
To allow for more sophisticated communication schemes between computation nodes in distributed systems we utilise the concept of Futures to wrap direct communication.

To show that PArrows have similar expressive power as existing parallel languages, we implement several skeletons and four benchmarks. 
Benchmarks show that our framework does not induce any notable performance overhead. We conclude that Arrows have considerable potential for composing parallel programs, and for producing programs that can execute on multiple parallel language implementations. 

\olcomment{Summarise conclusions}

\mbcomment{Jedes Kapitel soll einmal ins Abstract. Conclusions sollen mit ins Abstract}
\end{abstract}
