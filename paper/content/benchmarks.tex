
%
\section{Performance results}
\label{sec:benchmarks}

In the following section, we describe the benchmarks of our parallel DSL and algorithmic skeletons conducted.
We start by explaining the hardware and software stack and also elaborate on which benchmarks programs were used
and also which parallel Haskell were used in which setting. Before we go into detail on the benchmarks we
also address the issue of poor performance gains when using hyper-threading and our reasoning behind not including
hyper-threaded cores in our benchmarks. Finally, we show that PArrows hold up in terms of performance when compared to the original parallel Haskells used as backends in this paper, starting with the shared-memory variants (GpH, |Par| Monad and Eden CP) and concluding with Eden as a distributed backend.


\newcommand{\rmtest}{Rabin--Miller test\xspace}
\newcommand{\sudokutest}{Sudoku\xspace}
\newcommand{\jacobitest}{Jacobi sum test\xspace}
\newcommand{\torustest}{Gentleman\xspace}
\newlength{\plotwidthSMP}
\setlength{\plotwidthSMP}{0.39\textwidth}
\newlength{\plotwidthDist}
\setlength{\plotwidthDist}{0.6\textwidth}


\newcommand{\performanceplot}[7]{
\begin{tikzpicture}
\begin{axis}[title={#1},
title style={align=center},
scale only axis, width=#7,
xlabel=Threads,
%xtick=data,
xtick distance=#4,
ylabel=Time (s),
ylabel near ticks,
minor tick num=2,
grid=major,
legend entries={#2},
legend style={at={(0.99,0.99)},anchor=north east},
max space between ticks=50pt,
grid style={line width=.1pt, draw=gray!10},
major grid style={line width=.2pt,draw=gray!50},
xmin=-1,
xmax=#6]
#5
\end{axis}
\end{tikzpicture}
}

\newcommand{\performancediffplot}[8]{
\begin{tikzpicture}
\begin{axis}[title={#1},
title style={align=center},
scale only axis, width=#8,
xlabel=Threads,
%xtick=data,
ytick distance=#6,
xtick distance=#4,
minor tick num=9,
ylabel=Absolute time difference (s),
ylabel near ticks,
grid=both,
legend entries={#2},
legend style={at={(0.99,0.99)},anchor=north east},
max space between ticks=50pt,
grid style={line width=.1pt, draw=gray!10},
major grid style={line width=.2pt,draw=gray!50},
xmin=-1,
xmax=#7]
#5
\end{axis}
\end{tikzpicture}
}

\newcommand{\speedupplot}[7]{
\begin{tikzpicture}
\begin{axis}[title={#1},
title style={align=center},
scale only axis, width=#7,
xlabel=Threads,
%xtick=data,
%ytick=data,
xtick distance=#4,
ytick distance=#4,
ylabel=Speedup,
ylabel near ticks,
grid=major,
legend entries={linear, #2},
legend style={at={(0.01,0.99)},anchor=north west},
max space between ticks=50pt,
grid style={line width=.1pt, draw=gray!10},
major grid style={line width=.2pt,draw=gray!50},
ymin=-1,
xmin=-1,
ymax=#6,
xmax=#6]
\addplot [domain=0:#3, no markers,dotted,thick]{x};
#5
\end{axis}
\end{tikzpicture}
}

\newcommand{\speedupdiffplot}[7]{
\begin{tikzpicture}
\begin{axis}[title={#1},
title style={align=center},
scale only axis, width=#7,
xlabel=Threads,
%xtick=data,
xtick distance=#4,
ytick distance=0.5,
ylabel=Absolute speedup difference,
ylabel near ticks,
grid=major,
legend entries={#2},
legend style={at={(0.99,0.99)},anchor=north east},
max space between ticks=50pt,
grid style={line width=.1pt, draw=gray!10},
major grid style={line width=.2pt,draw=gray!50},
xmin=-1,
xmax=#6]
#5
\end{axis}
\end{tikzpicture}
}

\subsection{Hardware and Software used}

Benchmarks were run both in a shared and in a distributed
memory setting. All benchmarks were done on the Glasgow GPG Beowulf cluster, consisting of
16 machines with 2 Intel\SymbReg~Xeon\SymbReg~E5-2640 v2 and 64 GB of DDR3 RAM each. Each processor has 8 cores and 16 (hyper-threaded) threads with a base frequency of 2 GHz and a turbo frequency of 2.50 GHz. This results in a total of 256 cores and 512 threads for the whole cluster. The operating system was Ubuntu 14.04 LTS with Kernel 3.19.0-33. Non-surprisingly, we found that hyper-threaded 32 cores do not behave in the same manner as real 16 cores (numbers here for a single machine). We disregarded the hyper-threading ability in most of the cases.

Apart from Eden, all benchmarks and libaries were compiled with Stack's \footnote{see \url{https://www.haskellstack.org/}} lts-7.1 GHC compiler which is equivalent to a standard GHC 8.0.1 with the base package in version 4.9.0.0. Stack itself was used in version 1.3.2. For GpH in its Multicore variant we used the the parallel cabal package in version 3.2.1.0, while for the |Par| monad we used monad-par in version 0.3.4.8. For all Eden tests, we used its GHC-Eden compiler in version 7.8.2 together with OpenMPI 1.6.5.

Furtermore, all benchmarks were done with help of the bench \footnote{see \url{https://hackage.haskell.org/package/bench}} tool in version 1.0.2 which uses criterion (>=1.1.1.0 \&\& < 1.2) internally. All runtime data (mean runtime, max stddev, etc.) was collected with this tool if not mentioned otherwise.

We used a single node with 16 real cores as a shared memory testbed
and the whole grid with 256 real cores as a device to test our
distributed memory software.

\subsection{Benchmarks}

We used multiple tests that originated from different
sources. Most of them are parallel mathematical computations, initially
implemented in Eden. Table~\ref{tab:benches} summarises.

\begin{table}
\centering
%% something was wrong with separators in table
\renewcommand{\tabcolsep}{0.5em}
\begin{tabular}{lccll}
\toprule
Name & Area & Type & Origin & Source \\
\midrule
\rmtest & Mathematics & \ensuremath{\Varid{parMap}\mathbin{+}\Varid{reduce}} & Eden & \citet{Lobachev2012}\\
\jacobitest & Mathematics & \ensuremath{\Varid{workpool}\mathbin{+}\Varid{reduce}} & Eden & \citet{Lobachev2012}\\
\torustest & Mathematics & \ensuremath{\Varid{torus}} & Eden & \citet{Eden:SkeletonBookChapter02}\\
\sudokutest & Puzzle & \ensuremath{\Varid{parMap}} & \ensuremath{\Conid{Par}} Monad & \citet{par-monad} 
\tablefootnote{actual code from: http://community.haskell.org/\~simonmar/par-tutorial.pdf and https://github.com/simonmar/parconc-examples}\\
\bottomrule
\end{tabular}
\label{tab:benches}
\caption{The benchmarks we use in this paper.}
\end{table}

\rmtest is a probabilistic primality test that iterates multiple (here: 32--256)
\enquote{subtests}. Should a subtest fail, the input is definitely not a
prime. If all $n$ subtest pass, the input is composite with the
probability of $1/4^{n}$. 

Jacobi sum test or APRCL is also a primality test, that however,
guarantees the correctness of the result. It is probabilistic in the
sense that its run time is not certain. Unlike \rmtest, the subtests
of Jacobi sum test have very different durations. \citet{lobachev-phd}
discusses some optimisations of parallel APRCL. Generic parallel
implementations of \rmtest and APRCL were presented in \citet{Lobachev2012}.

\enquote{Gentleman} is a standard Eden test program, developed
for their \ensuremath{\Varid{torus}} skeleton. It implements a Gentleman's algorithm for parallel matrix
multiplication \citep{Gentleman1978}. We ported an Eden based version \citep{Eden:SkeletonBookChapter02} to PArrows.

A~parallel Sudoku solver was used by \citet{par-monad} to compare \ensuremath{\Conid{Par}} Monad
to GpH, and we ported it to PArrows.

%\bestAndWorstBenchmarks

\begin{table}[]
\centering
\caption{Overhead in the shared memory benchmarks. Bold marks values
  in favour of PArrows. Overhead is unit-less, multiply by 100 to
  obtain percent. Runtime is in seconds.}
\label{tab:meanOverheadSharedMemory}
\centering
\begingroup\catcode`"=9
\begin{tabular}{C{4cm} C{2cm} C{1.7cm} C{2.3cm} C{1.7cm}}
	\thead{Benchmark} & \thead{Base}             & \thead{Mean of \\ mean \\overheads} & \thead{Maximum \\ normalised \\ stdDev} & \thead{Runtime for \\ 16 cores} \\ \hline \\
	\multirow{3}{*}{\sudokutest 1000}
	\csvreader[head to column names]{benchmarks/sudoku-sm/bestAndWorstSudoku-1000.csv}{}{& \vs & \meanOverhead & \maxStdDevForOverhead & \runtimeMaxCores \\}
	\\ \hline \\
	\multirow{1}{*}{\torustest 512}
	\csvreader[head to column names]{benchmarks/torus-sm/bestAndWorstTorusSM-512.csv}{}{& \vs & \meanOverhead & \maxStdDevForOverhead & \runtimeMaxCores \\}
	\\ \hline \\
	\multirow{3}{*}{\rmtest 11213 32}
	\csvreader[head to column names]{benchmarks/sm-rm/bestAndWorstRMSM-11213-32.csv}{}{& \vs & \meanOverhead & \maxStdDevForOverhead & \runtimeMaxCores \\}
	\\ \hline \\
	\multirow{3}{*}{\rmtest 11213 64}
	\csvreader[head to column names]{benchmarks/sm-rm/bestAndWorstRMSM-11213-64.csv}{}{& \vs & \meanOverhead & \maxStdDevForOverhead & \runtimeMaxCores \\}
\end{tabular}
\endgroup
\end{table}

\begin{table}[]
\centering
\caption{Overhead in the distributed memory benchmarks. Bold marks values
  in favour of PArrows. Overhead is unit-less, multiply by 100 to
  obtain percent. Runtime is in seconds.}
\olcomment{Vielleicht gleich durchmultiplizieren und \% angeben? Nicht
zu viele leere Zeilen in der Tabelle?}
\label{tab:meanOverHeadDistributedMemory}
\centering
\begingroup\catcode`"=9
\begin{tabular}{C{4cm} C{2cm} C{1.7cm} C{2.3cm} C{1.7cm}}
	\thead{Benchmark} & \thead{Base}             & \thead{Mean of \\ mean \\overheads} & \thead{Maximum \\ normalised \\ stdDev} & \thead{Runtime for \\ 256 cores} \\ \hline \\
	\multirow{1}{*}{\torustest 4096}
	\csvreader[head to column names]{benchmarks/distributed-torus/bestAndWorstTorus-4096.csv}{}{& \vs & \meanOverhead & \maxStdDevForOverhead & \runtimeMaxCores \\}
	\\ \hline \\
  	\multirow{1}{*}{\rmtest 44497 256}
	\csvreader[head to column names]{benchmarks/distributed-rm/bestAndWorstRM-44497-256.csv}{}{& \vs & \meanOverhead & \maxStdDevForOverhead & \runtimeMaxCores \\}
	\\ \hline \\
	\multirow{1}{*}{\jacobitest 3217}
	\csvreader[head to column names]{benchmarks/distributed-jacobi/bestAndWorstJacobi-3217.csv}{}{& \vs & \meanOverhead & \maxStdDevForOverhead & \runtimeMaxCores \\}
\end{tabular}
\endgroup
\end{table}


\subsection{What parallel Haskells run where}

The \ensuremath{\Conid{Par}} monad and GpH -- in its multicore version \cite{Marlow2009} --  can be executed on a shared
memory machines only. Although GpH is available on distributed memory
clusters, and newer distributed memory Haskells such as HdpH exist,
current support of distributed memory in PArrows is limited to
Eden. We used the MPI backend of Eden in a distributed memory
setting. However, for shared memory Eden features a ``CP'' backend
that merely copies the memory blocks between distributed heaps. In
this mode, Eden still operates in the ``nothing shared'' setting, but
is adapted better to multicore machines. We label this version of Eden
in the plots as ``Eden~CP''.



\subsection{Effect of hyper-threading}

In preliminary tests, the PArrows version of \rmtest on a single node of the Glasgow cluster
showed almost linear speedup on up to 16 shared-memory cores (Fig.~\ref{fig:bench-rm-sm}). The speedup
of 64-task PArrows/Eden at 16 real cores version was 13.65 giving a parallel
efficiency of 85.3\%. However, if we increased the number of
requested cores to 32 -- \ie if we use hyper-threading on 16 real
cores -- the speedup did not increase that well. It was merely 15.99
for 32 tasks with PArrows/Eden. This was worse for other backends.  As
for 64 tasks, we obtained a speedup of 16.12 with PArrows/Eden at 32
hyper-threaded cores and only 13.55 with PArrows/GpH. The parallel eficiencies were 50.4\% and 42.3\%, respectively. The Eden
version used here was Eden~CP, the \enquote{share nothing} SMP build.

In the distributed memory setting the same effect ensues. We obtain
plummeting speedup of 124.31 at 512 hyper-threaded cores, whereas it was
213.172 for 256 real cores. Apparently, hyper-threading in the Glasgow
cluster fails to execute two parallel Haskell processes with full-fledged
parallelism.

Hence, the performance measurements in Figs.~\ref{fig:bench-rm-sm}--\ref{fig:sudokuSMBenchmark} and table~\ref{tab:bestAndWorstResults} use only real cores as the purpose of this paper is to show the performance of PArrows and not to investigate parallel performance with hyper-threading in use.


% rm 11213 32 32-sm speedup eden: 15.993037587283924
% -"- multi: 15.09948017762912
% -"- par: 14.909092857846693

% -"- 64 32-sm speedup eden: 16.118040224478424
% -"- multi: 13.545304115702333
% -"- par: 15.155709987503396

\subsection{Benchmark results}

The difference between, say, PArrows with \ensuremath{\Conid{Par}} Monad backend and a
genuine \ensuremath{\Conid{Par}}
Monad benchmark is very small. To give an example, it is $0.4\seconds$ in favour of PArrows for 16 cores ($10.8\seconds$ vs. $11.2\seconds$) and $-0.8\seconds$ in favour of the \ensuremath{\Conid{Par}} monad for 8 cores ($16.1\seconds$ vs. $16.9\seconds$) for
the Sudoku benchmark in the shared memory setting. It is almost invisible in speedup and
(non shown) run time plots. We thus show only the results for the
PArrows-enabled versions in the backend-comparison.

To show that PArrows induce very small overhead even in a distributed context (and in general as well), we compare the original Eden
versions of the benchmark to its PArrows-enabled counterpart in the \rmtest, \torustest and \jacobitest benchmarks. We plot execution time differences between measurements for
PArrows and the corresponding backend in a separate plot
(Figs.~\ref{fig:bench-rm-dist}--\ref{fig:torusBenchmark}). As an example, the differences range in
about $0.5\seconds$ for the execution time of $46\seconds$ on 256 cores
for distributed \rmtest with PArrows and Eden. For these comparisons, the plots show absolute
time differences that are not relative \wrt the total execution time.
Furthermore, the error bars ends were computed from point-wise maximum of both standard
deviations from both measurements for PArrows and non-PArrows
versions. These are the values provided by the \ensuremath{\Varid{bench}} package that we
used for bench-marking. We call a difference between two versions
significant when the border of the error bar of absolute time
difference is above or below zero. In other words: the time
difference is significant if it is outside of the measurement error.

\subsubsection{\rmtest}\label{sec:rmtest}

\newcommand{\performanceSkelRMSM}[2]{
\performanceplot{Parallel run time of \rmtest \enquote{#2}}{Eden CP, GpH, \ensuremath{\Conid{Par}} Monad}{16}{4}{
\addplot+ [very thick] table [scatter, x="nCores", y="time", col sep=comma, mark=none,
smooth]{benchmarks/sm-rm/bench-sm-rm.bench.skelrm-parr-eden-cp-#1-#2.csv};
\addplot+ [very thick] table [scatter, x="nCores", y="time", col sep=comma, mark=none,
smooth]{benchmarks/sm-rm/bench-sm-rm.bench.skelrm-parr-mult-#1-#2.csv};
\addplot+ [very thick] table [scatter, x="nCores", y="time", col sep=comma, mark=none,
smooth]{benchmarks/sm-rm/bench-sm-rm.bench.skelrm-parr-par-#1-#2.csv};
}{17}{\plotwidthSMP}
}

\newcommand{\speedupSkelRMSM}[2]{
\speedupplot{Speedup of \rmtest \enquote{#2}}{Eden CP, GpH, \ensuremath{\Conid{Par}} Monad}{16}{4}{
\addplot+ [very thick] table [scatter, x="nCores", y="speedup", col sep=comma, mark=none,
smooth]{benchmarks/sm-rm/bench-sm-rm.bench.skelrm-parr-eden-cp-#1-#2.csv};
\addplot+ [very thick] table [scatter, x="nCores", y="speedup", col sep=comma, mark=none,
smooth]{benchmarks/sm-rm/bench-sm-rm.bench.skelrm-parr-mult-#1-#2.csv};
\addplot+ [very thick] table [scatter, x="nCores", y="speedup", col sep=comma, mark=none,
smooth]{benchmarks/sm-rm/bench-sm-rm.bench.skelrm-parr-par-#1-#2.csv};
}{17}{\plotwidthSMP}
}

\newcommand{\speedupSkelRMDist}[4]{
\speedupplot{Speedup of \rmtest \enquote{#1 #2}}{PArrows}{256}{#3}{
% \addplot [mark=*,very thick] table [scatter, x="nCores", y="speedup", col sep=comma, mark=none,
% smooth]{benchmarks/distributed-rm/bench-distributed.bench.skelrm-parrows-11213-#2.csv};
\addplot [mark=*,very thick,blue] table [scatter, x="nCores", y="speedup", col sep=comma, mark=none,
smooth]{benchmarks/distributed-rm/bench-distributed.bench.skelrm-parrows-#1-#2.csv};
% \addplot table [scatter, x="nCores", y="speedup", col sep=comma, mark=none,
% smooth]{benchmarks/distributed-rm/bench-distributed.bench.skelrm-eden-#1-#2.csv};
}{#4}{\plotwidthDist}
}

\newcommand{\performanceSkelRMDistDiff}[5]{
\performancediffplot{Run time differences\\for \rmtest \enquote{#1 #2}}{(Eden $-$ PArrows)}{256}{#3}{
\addplot+[mark=*,very thick,error bars/.cd,
    y dir=both,y explicit] table [x="nCores", y="time", y error="max stddev", col sep=comma, mark=dots,
smooth]{benchmarks/distributed-rm/#1-#2-diff.csv};
}{#4}{#5}{\plotwidthDist}
}

\begin{figure}
%\centering
%\performanceSkelRMSM{11213}{64}\hfill%
{\speedupSkelRMSM{11213}{32}}\hfill%
{\speedupSkelRMSM{11213}{64}}
\caption{Relative speedup of \rmtest on a multicore machine. We used the same PArrows-based implementation with
  different backends on the same hardware. Measurements were performed on a single node of the Glasgow
  cluster; it has 16 real cores. Input was $2^{11213}-1$, we used 32 (left) or 64 (right)
  tasks. The
  closer to linear speedup the better.}
\label{fig:bench-rm-sm}
\end{figure}

\begin{figure}
\centering
%\performanceSkelRMDist{44497}{256}{32,64,128,256,512}{544}
%
{\speedupSkelRMDist{44497}{256}{32}{272}\label{subfig:rm-dist-a}}%
%\hfill%
{\performanceSkelRMDistDiff{44497}{256}{32}{0.5}{272}\label{subfig:rm-dist-b}}
\caption{Parallel performance of the \rmtest on the Glasgow cluster
  consisting of 256 cores. Input was $2^{44497}-1$, we used 256
  tasks. The top plot shows absolute speedup in a distributed memory setting. The
  closer to linear speedup the better. Time
  (and hence speedup) measurements for PArrows with Eden backend and
  Eden almost coincide. Hence, bottom plot shows
absolute time differences for this benchmark. The
higher the value, the better for PArrows\olcomment{CHECKME}.}
\label{fig:bench-rm-dist}
\end{figure}

%\olcomment{THE ACTUAL TEXT IS MISSING. What do we see in the plots?
%  Why is it good?}
The multicore version of our parallel \rmtest benchmark is depicted in
Figure~\ref{fig:bench-rm-sm}. We executed the test with 32 and 64
tasks. The plot shows the PArrows-enabled versions with corresponding backends.
The performance of PArrows/Eden~CP in shared memory is slightly better than
for SMP variants such as PArrows/GpH and PArrows/\ensuremath{\Conid{Par}}
Monad but most of the time the performance is still comparable with the GpH backend performing slightly worse than the other two in terms of speedup.

Comparing the PArrows version of the \rmtest with the original from Eden with the MPI backend in a distributed memory setting, we see an almost linear speedup of
\rmtest with 256 tasks and input $2^{44497}-1$ in both versions. The sequential run time
was computed as the mean of three consecutive executions on a single
core---the single run took two hours 43 minutes. The difference between
PArrows/Eden and Eden almost always lies inside the error bar of
the measurement.

%As the PArrows version uses Eden in the backend, these numbers suggest that there is no real performance difference between using PArrows or Eden for this task as they trade blows in this benchmark. Additionally, PArrows with an Eden-based backend performing better than what it is based upon suggests that any difference in runtime between the two is more of an anomaly than a real difference.

\subsubsection{\jacobitest}

Continuing, the results of the \jacobitest (MPI only) in Fig.~\ref{fig:bench-jacobi-dist} are as follows:
The program does not seem to scale well beyond 64 threads with input $2^{3217}-1$. We once again compare the Eden version with the PArrrows version in Fig.~\ref{fig:bench-jacobi-dist}. We see similar behaviour to the MPI version of the \rmtest: The difference once again almost always remains within the bounds of the error bar of the measurement.

Because of the bad scaling behaviour for $2^{3217}-1$ beyond 64 threads, we also ran tests with input $2^{4253}-1$. Because of the long running time, we could do this only for the 128 and 256 threads. Therefore we do not show these results in Fig.~\ref{fig:bench-jacobi-dist} as we cannot properly compute a speedup without results for 1 thread. Nonetheless, for 128 threads, the benchmark took a mean of 9192.9\seconds, while with 256 threads, it only took a mean of 1649.1\seconds. This means that the 128 thread version ran more than 5.5 times slower than the 256 one, which suggests an IO limitation for this big input that is somewhat mitigated by adding more cores. However it still proves that PArrows continue to scale, even if not perfectly for this test program.
Comparing PArrows with Eden, for the larger input of $2^{4253}-1$, we see slightly bigger differences between Eden and PArrows: The differences for 128 and 256 core sare $-1005.18\seconds$ and $-94.33\seconds$ in favour of Eden, respectively. This maximum of $12.27\%$ slower PArrows runtime, however, was still in the (quite big) error bar of our measurements.

\newcommand{\speedupJacobiDist}[5]{
\speedupplot{Speedup of \jacobitest \enquote{#2} vs simulated \enquote{#5}}{PArrows #2, simulated PArrows #5}{256}{#3}{
% \addplot [mark=*,very thick] table [scatter, x="nCores", y="speedup", col sep=comma, mark=none,
% smooth]{benchmarks/distributed-rm/bench-distributed.bench.skelrm-parrows-11213-#2.csv};
\addplot [mark=*,very thick,blue] table [scatter, x="nCores", y="speedup", col sep=comma, mark=none,
smooth]{benchmarks/distributed-jacobi/bench-jacobi.bench.jacobi-parr-#1-#2.csv};
\addplot [mark=x,very thick,red] table [scatter, x="nCores", y="speedup"/, col sep=comma, mark=none,
smooth]{benchmarks/distributed-jacobi/bench-jacobi.bench.jacobi-parr-#1-#5.csv};
% \addplot table [scatter, x="nCores", y="speedup", col sep=comma, mark=none,
% smooth]{benchmarks/distributed-rm/bench-distributed.bench.skelrm-eden-#1-#2.csv};
}{#4}{\plotwidthDist}
}

\newcommand{\performanceJacobiDistDiff}[5]{
\performancediffplot{Run time differences\\for \jacobitest \enquote{#2}}{(Eden $-$ PArrows) #2}{256}{#3}{
\addplot+[mark=*,very thick,error bars/.cd,
    y dir=both,y explicit] table [x="nCores", y="time", y error="max stddev", col sep=comma, mark=dots,
smooth]{benchmarks/distributed-jacobi/#1-#2-diff.csv};
}{#4}{#5}{\plotwidthDist}
}

\begin{figure}
\centering
%\performanceSkelRMDist{44497}{256}{32,64,128,256,512}{544}
%
{\speedupJacobiDist{3}{3217}{32}{272}{4253}\label{subfig:jacobi-dist-a}}%
%\hfill%
{\performanceJacobiDistDiff{3}{3217}{32}{5}{272}\label{subfig:jacobi-dist-b}}
\caption{Parallel performance of the \jacobitest on the Glasgow cluster
  consisting of 256 cores. Input was $2^{3217}-1$, we used 256
  tasks. The top plot shows relative speedup in a distributed memory setting compared to a simulated speedup for input $2^{4253}-1$. The
  closer to linear speedup the better. Time
  (and hence speedup) measurements for PArrows with Eden backend and
  Eden almost coincide. Hence, bottom plot shows
absolute time differences for this benchmark. The
higher the value, the better for PArrows\olcomment{CHECKME}.}
\label{fig:bench-jacobi-dist}
\end{figure}

\subsubsection{\torustest}

Next is the \torustest benchmark. The results of the comparison of vanilla Eden to our PArrows-based version can be found in Fig.~\ref{fig:torusBenchmark}. We see that the benchmark scales quite well with more cores until 64 cores. For $>=96$ cores, we still have considerable speedup, but with less slope. We also prove that the difference between the Eden and PArrows version are only marginal with PArrows only being a maximum of $1.7\%$ slower -- for 160 cores -- when outside of the error bar.  %The difference between PArrows and Eden is only significant for 16 and 64 cores where it ran 1.7\% and 2.7\% slower which corresponds to a real-time difference of 0.12s and 0.13s. For 256 cores PArrows performed 0.2\% slower which corresponds to 0.01s overhead.

\newcommand{\speedupTorusDist}[3]{
\speedupplot{Speedup of \torustest \enquote{#1}}{PArrows}{256}{#2}{
\addplot [mark=*,very thick,blue] table [scatter, x="nCores", y="speedup", col sep=comma, mark=none,
smooth]{benchmarks/distributed-torus/bench-torus-distributed.bench.torus-matrix-parrows-#1.csv};
}{#3}{\plotwidthDist}
}


\newcommand{\performanceTorusDistDiff}[4]{
\performancediffplot{Run time differences\\for \torustest \enquote{#1}}{(Eden $-$ PArrows)}{256}{#2}{
\addplot+[mark=*,very thick,error bars/.cd,
    y dir=both,y explicit] table [x="nCores", y="time", y error="max stddev", col sep=comma, mark=dots,
smooth]{benchmarks/distributed-torus/#1-diff.csv};
}{#3}{#4}{\plotwidthDist}
}

\begin{figure}
\centering
%\performanceSkelRMDist{44497}{256}{32,64,128,256,512}{544}
%
{\speedupTorusDist{4096}{32}{272}\label{subfig:speedupTorusDist}}%
%\hfill%
{\performanceTorusDistDiff{4096}{32}{0.5}{272}\label{subfig:performancetorusDistDiff}}
\caption{Parallel performance of \torustest on the Glasgow cluster
  consisting of 256 cores. Input was a matrix size of $4096$. The top plot shows absolute speedup in a distributed memory setting. The
  closer to linear speedup the better. Time
  (and hence speedup) measurements for PArrows with Eden backend and
  Eden almost coincide. Hence, bottom plot shows
absolute time differences for this benchmark. The
higher the value, the better for PArrows\olcomment{CHECKME}.}
\label{fig:torusBenchmark}
\end{figure}


\subsubsection{\sudokutest}

As the last benchmark in this paper we present the \sudokutest in Fig.~\ref{fig:sudokuSMBenchmark} running in a shared memory setting. Here we see all three SM backends performing similarly again like in the \rmtest SM benchmarks in Figs.~\ref{fig:sudokuSMBenchmark} and \ref{fig:sudokuSMBenchmark16000}. However, we notice that the GpH backend seems to choke on a bigger input (Fig.~\ref{fig:sudokuSMBenchmark16000}). This is due to the benchmark only using \ensuremath{\Varid{parMap}} instead of a chunking variant -- however we did not change that for simplicity's sake. This issue is reflected by debug output which shows that of 16000 sparks being created (one for each Sudoku) only 8365 were converted (executed) with the rest (7635) overflowing the runtime spark pool. Another remarkable finding is that the Eden backend seems to lag behind for $\leq$16 threads, but manages to pull ahead noticeably with all 32 threads of the system in use.


\newcommand{\performanceSudokuSM}[1]{
\performanceplot{Parallel run time of \sudokutest \enquote{#1}}{linear speedup, Eden CP, GpH, \ensuremath{\Conid{Par}} Monad}{16}{4}{
\addplot [no markers,dotted,thick] table [scatter, x="nCores", y="time", col sep=comma, mark=none,
smooth] {benchmarks/sudoku-sm/bench-sudoku-sm.bench.fake-linear-sudoku-sudoku17.#1.txt.csv};
\addplot+ [very thick] table [scatter, x="nCores", y="time", col sep=comma, mark=none,
smooth]{benchmarks/sudoku-sm/bench-sudoku-sm.bench.parrows-sudoku-parmap-eden-sudoku17.#1.txt.csv};
\addplot+ [very thick] table [scatter, x="nCores", y="time", col sep=comma, mark=none,
smooth]{benchmarks/sudoku-sm/bench-sudoku-sm.bench.parrows-sudoku-parmap-mult-sudoku17.#1.txt.csv};
\addplot+ [very thick] table [scatter, x="nCores", y="time", col sep=comma, mark=none,
smooth]{benchmarks/sudoku-sm/bench-sudoku-sm.bench.parrows-sudoku-parmap-par-sudoku17.#1.txt.csv};
%\addplot+ [very thick] table [scatter, x="nCores", y="time", col sep=comma, mark=none,
%smooth]{benchmarks/sudoku-sm/bench-sudoku-sm.bench.parmonad-sudoku-sudoku17.#1.txt.csv};
}{17}{\plotwidthSMP}
}

\newcommand{\speedupSudokuSM}[1]{
\speedupplot{Parallel speedup of \sudokutest \enquote{#1}}{Eden CP, GpH, \ensuremath{\Conid{Par}} Monad}{16}{4}{
\addplot+ [very thick] table [scatter, x="nCores", y="speedup", col sep=comma, mark=none,
smooth]{benchmarks/sudoku-sm/bench-sudoku-sm.bench.parrows-sudoku-parmap-eden-sudoku17.#1.txt.csv};
\addplot+ [very thick] table [scatter, x="nCores", y="speedup", col sep=comma, mark=none,
smooth]{benchmarks/sudoku-sm/bench-sudoku-sm.bench.parrows-sudoku-parmap-mult-sudoku17.#1.txt.csv};
\addplot+ [very thick] table [scatter, x="nCores", y="speedup", col sep=comma, mark=none,
smooth]{benchmarks/sudoku-sm/bench-sudoku-sm.bench.parrows-sudoku-parmap-par-sudoku17.#1.txt.csv};
%\addplot+ [very thick] table [scatter, x="nCores", y="speedup", col sep=comma, mark=none,
%smooth]{benchmarks/sudoku-sm/bench-sudoku-sm.bench.parmonad-sudoku-sudoku17.#1.txt.csv};
}{17}{\plotwidthSMP}
}

\begin{figure}
\centering
%\performanceSkelRMDist{44497}{256}{32,64,128,256,512}{544}
%
{\speedupSudokuSM{1000}\label{subfig:speedupSudokuSM}}%
%\hfill%
{\performanceSudokuSM{1000}\label{subfig:performanceSudokuSM}}
\caption{Absolute speedup of \sudokutest on a multicore machine. We used the same PArrows-based implementation with
  different backends on the same hardware and the \ensuremath{\Varid{parMap}} version from the \ensuremath{\Conid{Par}} Monad examples. Measurements were performed on a single node of the Glasgow
  cluster; it has 16 real cores and 32 threads. Input was a file of $1000$ Sudokus. The
  closer to linear speedup the better.}
\label{fig:sudokuSMBenchmark}
\end{figure}

\begin{figure}
\centering
%\performanceSkelRMDist{44497}{256}{32,64,128,256,512}{544}
%
{\speedupSudokuSM{16000}\label{subfig:speedupSudokuSM16000}}%
%\hfill%
{\performanceSudokuSM{16000}\label{subfig:performanceSudokuSM16000}}
\caption{Absolute speedup of \sudokutest on a multicore machine. We used the same PArrows-based implementation with
  different backends on the same hardware and the \ensuremath{\Varid{parMap}} version from the \ensuremath{\Conid{Par}} Monad examples. Measurements were performed on a single node of the Glasgow
  cluster; it has 16 real cores and 32 threads. Input was a file of $16000$ Sudokus. The
  closer to linear speedup the better. The GpH version shows signs of choking with too many sparks being created.}
\label{fig:sudokuSMBenchmark16000}
\end{figure}

\subsection{Byline}

In the shared memory setting we naturally find that the backends perform differently. Furthermore, in our distributed memory tests on the full grid the difference between PArrows and Eden was almost always below the error margin if PArrows performed worse. Even the biggest difference of 12.27\% in the \jacobitest for input $2^{4253}-1$ and 128 cores remained in line with these findings. The only exception to this was the \torustest  with only a $1.7\%$ difference in favour of Eden which is only a marginal slowdown in our opinion.
