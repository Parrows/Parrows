
%
\section{Performance results and discussion}
\label{sec:benchmarks}

In the following section, we describe the benchmarks we conducted of our parallel DSL and algorithmic skeletons.
We start by explaining the hardware and software stack and also elaborate on which benchmarks programs and parallel Haskells were used in which setting and why. Before we go into detail on the benchmarks we
also shortly address hyper-threading and why we do not use it in our benchmarks. Finally, we show that PArrows hold up in terms of performance when compared to the original parallel Haskells used as backends in this paper, starting with the shared-memory variants (GpH, |Par| Monad and Eden CP) and concluding with Eden as a distributed backend.


\newcommand{\rmtest}{Rabin--Miller test\xspace}
\newcommand{\sudokutest}{Sudoku\xspace}
\newcommand{\jacobitest}{Jacobi sum test\xspace}
\newcommand{\torustest}{Gentleman\xspace}
\newlength{\plotwidthSMP}
\setlength{\plotwidthSMP}{0.39\textwidth}
\newlength{\plotwidthDist}
\setlength{\plotwidthDist}{0.6\textwidth}

\newcommand{\benchmarkDir}{benchmarks}

\newcommand{\speedupplot}[8]{
\begin{tikzpicture}
\begin{axis}[title={#1},
title style={align=center},
scale only axis, width=#7,
xlabel=Threads,
%xtick=data,
%ytick=data,
xtick distance=#4,
ytick distance=#4,
ylabel=Speedup,
ylabel near ticks,
grid=major,
legend entries={linear, #2},
legend style={at={(0.01,0.99)},anchor=north west},
max space between ticks=50pt,
grid style={line width=.1pt, draw=gray!10},
major grid style={line width=.2pt,draw=gray!50},
ymin=-1,
xmin=-1,
ymax=#8,
xmax=#6]
\addplot [domain=0:#3, no markers,dotted,thick]{x};
#5
\end{axis}
\end{tikzpicture}
}

\subsection{Preliminaries}

\subsubsection{Hardware and Software used}

Benchmarks were run both in a shared and in a distributed
memory setting. All benchmarks were done on the Glasgow GPG Beowulf cluster, consisting of
16 machines with 2 Intel\SymbReg~Xeon\SymbReg~E5-2640 v2 and 64 GB of DDR3 RAM each. Each processor has 8 cores and 16 (hyper-threaded) threads with a base frequency of 2 GHz and a turbo frequency of 2.50 GHz. This results in a total of 256 cores and 512 threads for the whole cluster. The operating system was Ubuntu 14.04 LTS with Kernel 3.19.0-33. Non-surprisingly, we found that hyper-threaded 32 cores do not behave in the same manner as real 16 cores (numbers here for a single machine). We disregarded the hyper-threading ability in most of the cases.

Apart from Eden, all benchmarks and libraries were compiled with Stack's\footnote{see \url{https://www.haskellstack.org/}} lts-7.1 GHC compiler which is equivalent to a standard GHC 8.0.1 with the base package in version 4.9.0.0. Stack itself was used in version 1.3.2. For GpH in its Multicore variant we used the the parallel package in version 3.2.1.0\footnote{see \url{https://hackage.haskell.org/package/parallel-3.2.1.0}}, while for the |Par| Monad we used monad-par in version 0.3.4.8\footnote{see \url{https://hackage.haskell.org/package/monad-par-0.3.4.8}}. For all Eden tests, we used its GHC-Eden compiler in version 7.8.2\footnote{see \url{http://www.mathematik.uni-marburg.de/~eden/?content=build_eden_7_&navi=build}} together with OpenMPI 1.6.5\footnote{see \url{https://www.open-mpi.org/software/ompi/v1.6/}}.

Furthermore, all benchmarks were done with help of the bench\footnote{see \url{https://hackage.haskell.org/package/bench}} tool in version 1.0.2 which uses criterion (>=1.1.1.0 \&\& < 1.2)\footnote{see \url{https://hackage.haskell.org/package/criterion-1.1.1.0}} internally. All runtime data (mean runtime, max stddev, etc.) was collected with this tool if not mentioned otherwise.

We used a single node with 16 real cores as a shared memory test-bed
and the whole grid with 256 real cores as a device to test our
distributed memory software.

\subsubsection{Benchmarks}

We used multiple tests that originated from different
sources. Most of them are parallel mathematical computations, initially
implemented in Eden. Table~\ref{tab:benches} summarises.

\begin{table}
\centering
\caption{The benchmarks we use in this paper.}
\label{tab:benches}
%% something was wrong with separators in table
\renewcommand{\tabcolsep}{0.5em}
\begin{tabular}{lccll}
\toprule
Name & Area & Type & Origin & Source \\
\midrule
\rmtest & Mathematics & \ensuremath{\Varid{parMap}\mathbin{+}\Varid{reduce}} & Eden & \citet{Lobachev2012}\\
\jacobitest & Mathematics & \ensuremath{\Varid{workpool}\mathbin{+}\Varid{reduce}} & Eden & \citet{Lobachev2012}\\
\torustest & Mathematics & \ensuremath{\Varid{torus}} & Eden & \citet{Eden:SkeletonBookChapter02}\\
\sudokutest & Puzzle & \ensuremath{\Varid{parMap}} & \ensuremath{\Conid{Par}} Monad & \citet{par-monad}\tablefootnote{actual code from: \url{http://community.haskell.org/\~simonmar/par-tutorial.pdf and https://github.com/simonmar/parconc-examples}}\\
\bottomrule
\end{tabular}
\end{table}

\rmtest is a probabilistic primality test that iterates multiple (here: 32--256)
\enquote{subtests}. Should a subtest fail, the input is definitely not a
prime. If all $n$ subtest pass, the input is composite with the
probability of $1/4^{n}$. 

Jacobi sum test or APRCL is also a primality test, that however,
guarantees the correctness of the result. It is probabilistic in the
sense that its run time is not certain. Unlike \rmtest, the subtests
of Jacobi sum test have very different durations. \citet{lobachev-phd}
discusses some optimisations of parallel APRCL. Generic parallel
implementations of \rmtest and APRCL were presented in \citet{Lobachev2012}.

\enquote{Gentleman} is a standard Eden test program, developed
for their \ensuremath{\Varid{torus}} skeleton. It implements a Gentleman's algorithm for parallel matrix
multiplication \citep{Gentleman1978}. We ported an Eden based version \citep{Eden:SkeletonBookChapter02} to PArrows.

A~parallel Sudoku solver was used by \citet{par-monad} to compare \ensuremath{\Conid{Par}} Monad
to GpH, and we ported it to PArrows.

\subsubsection{What parallel Haskells run where}

The \ensuremath{\Conid{Par}} monad and GpH -- in its multicore version \cite{Marlow2009} --  can be executed on shared memory machines only.
Although GpH is available on distributed memory
clusters, and newer distributed memory Haskells such as HdpH exist,
current support of distributed memory in PArrows is limited to
Eden. We used the MPI backend of Eden in a distributed memory
setting. However, for shared memory Eden features a \enquote{CP} backend
that merely copies the memory blocks between distributed heaps. In
this mode, Eden still operates in the \enquote{nothing shared} setting, but
is adapted better to multicore machines. We call this version of Eden
\enquote{Eden~CP}.



\subsubsection{Effect of hyper-threading}

In preliminary tests, the PArrows version of \rmtest on a single node of the Glasgow cluster
showed almost linear speedup on up to 16 shared-memory cores (as supplementary materials show). The speedup
of 64-task PArrows/Eden at 16 real cores version was 13.65 giving a parallel
efficiency of 85.3\%. However, if we increased the number of
requested cores to 32 -- \ie if we use hyper-threading on 16 real
cores -- the speedup did not increase that well. It was merely 15.99
for 32 tasks with PArrows/Eden. This was worse for other backends.  As
for 64 tasks, we obtained a speedup of 16.12 with PArrows/Eden at 32
hyper-threaded cores and only 13.55 with PArrows/GpH. 

While this shows that hyper-threading can be of benefit in scenarios similar to the ones presented in the benchmarks, we only use real cores for the performance measurements in Section~\ref{sec:benchmarkResults} as the purpose of this paper is to show the performance of PArrows and not to investigate parallel behaviour with hyper-threading.


% rm 11213 32 32-sm speedup eden: 15.993037587283924
% -"- multi: 15.09948017762912
% -"- par: 14.909092857846693

% -"- 64 32-sm speedup eden: 16.118040224478424
% -"- multi: 13.545304115702333
% -"- par: 15.155709987503396

\subsection{Benchmark results}\label{sec:benchmarkResults}

In the following paragraphs we will go into detail on how well PArrows perform when compared to versions of the benchmark programs implemented with the original parallel Haskells that were used in the backends. We start with the definition of an Overhead to compare both PArrows-enabled and standard benchmark implementations. We continue comparing speedups and overheads for the shared memory backends and then study OpenMPI variants of the Eden-enabled backend as a representative of a distributed memory backend. We plot all speedup curves and all overhead values in the supplementary materials.

\subsubsection{Defining overhead}

We compared the overhead of our PArrows implementations to the original parallel Haskells. We basically implemented the test programs both in PArrows and in traditional style, benchmarked both and will now compare the mean overhead, i.e. the \textit{relative} difference between both benchmarks with same settings. The error margin of the time measurements, supplied by criterion package, is computed together with the mean overhead computation. 

Quite often the zero value lies in the error margin of the mean overhead. This means that even though we have measured some difference (against or even in favour of PArrows), it could be merely the error margin of the benchmarks and the difference might not be existent. We are mostly interested in the cases where above issue does not persist, we call them \emph{significant}. We often denote the error margin with $\pm$ after the mean overhead value.

\subsubsection{Shared memory}

\paragraph{Speedup}
The \rmtest benchmark showed almost linear speedup for both 32 and 64 tasks, the performance is slightly better in the latter case: 13.7 at 16 cores for input $2^{11213}-1$ and 64 tasks in the best case scenario with Eden CP. The performance of the \sudokutest benchmark merely reaches a speedup of 9.19 (GpH), 8.78 (Par Monad), 8.14 (Eden CP) for 16 cores and 1000 Sudokus. In contrast to Rabin--Miller, here the |GpH| backend seems to be the best of all, while Rabin--Miller profited most from Eden CP (i.e., Eden with direct memory copy) implementation of PArrows. Gentleman on shared memory has a plummeting speedup curve with GpH and |Par| Monad and logarithmically increasing speedup for the Eden-based version. The latter reached a speedup of 6.56 at 16 cores.

\paragraph{Overhead}

For the shared memory \rmtest benchmark, implemented with PArrows using Eden CP, GpH, and |Par| Monad backends, the overhead values are within single percents range, but also negative overhead (\ie PArrows are better) and larger error margins happen. To give a few examples, the overhead for Eden~CP with input value $2^{11213}-1$, 32 tasks, and 16 cores is $1.5\%$, but the error margin is around $5.2\%$! Same backend in the same setting with 64 tasks reaches $-0.2\%$ overhead, PArrows apparently fare better than Eden -- but the error margin of $1.9\%$ disallows this interpretation. We focus now on significant overhead values. To name a few: $0.41\%\; \pm 7\cdot 10^{-2}\%$ for Eden~CP and 64 tasks at 4 cores, $4.7\% \; \pm 0.72\%$ for GpH, 32 tasks, 8 cores, $0.34\% \; \pm 0.31\%$ for |Par| Monad at 4 cores with 64 tasks. The worst significant overhead was  GpH backend with $8\% \; \pm 6.9\%$ at 16 cores with 32 tasks and input value $2^{11213}-1$. In other words, we notice no major slow-down through PArrows here.

For Sudoku the situation is slightly different. There is a minimal significant ($-1.4\% \; \pm 1.2\%$ at 8 cores) speed improvement with PArrows Eden~CP version when compared with the base Eden CP benchmark. However, with increasing number of cores the error margin reaches zero again: $-1.6\% \; \pm 5.0\%$ at 16 cores. The |Par| Monad shows a similar development, \eg with $-1.95\% \; \pm 0.64\%$ at 8 cores. The GpH version shows both a significant speed \textit{improvement} from PArrows of $-4.2\% \; \pm 0.26\%$ (for 16 cores) and a minor overhead of $0.87\% \; \pm 0.70\%$ (4 cores).

The Gentleman multiplication with Eden CP shows a minor significant overhead of $2.6\% \; \pm 1.0\%$ at 8 cores and an insignificant improvement at 16 cores. Summarising, we observe a low (if significant at all) overhead, induced by PArrows in the shared memory setting.

\subsubsection{Distributed Memory}

\paragraph{Speedup}
The speedup of distributed memory Rabin--Miller benchmark with PArrows and distributed Eden backend showed an almost linear speedup sans some issues in the middle because of unfortunate task distribution. As seen in Fig.~\ref{fig:rabinMillerDistSpeedup}, we reached a speedup of 213.4 with PArrrows at 256 cores (vs. 207.7 with pure Eden). We managed to measure the speedup of Jacobi sum test for sufficient large inputs like $2^{4253}-1$ only in a massively distributed setting, because of memory limitations. We improved there from 9193 seconds (128 cores) to 1649 seconds (256 cores) in our PArrows version. A~scaled-down version with input $2^{3217}-1$ stagnates the speedup at about 11 for both PArrows and Eden for more than 64 cores. There is apparently not enough work for that many cores. The Gentleman test with input 4096 had an almost linear speedup first, then plummeted between 128 and 224 cores, and recovered at 256 cores with speedup of 129.

\begin{figure}[ht]
	\centering
	\speedupplot{Speedup of distributed \rmtest \enquote{44497 256}}{PArrows Eden}{256}{32}{
	\addplot+ [very thick] table [scatter, x="nCores", y="speedup", col sep=comma, mark=none,
	smooth]{\benchmarkDir/distributed-rm/bench-distributed.bench.skelrm-parrows-44497-256.csv};
	}{272}{\plotwidthDist}{272}
	\caption[Speedup distributed Rabin--Miller]{Speedup of the distributed \rmtest benchmark using PArrows with Eden backend.}
	\label{fig:rabinMillerDistSpeedup}
\end{figure}

\paragraph{Overhead}
We use our mean overhead quality measure and the notion of significance also for distributed memory benchmarks. The mean overhead of Rabin-Miller test in the distributed memory setting ranges from $0.29\%$ to $-2.8\%$ (last value in favour of PArrows), but these values are not significant with error margins $\pm 0.8\%$ and $\pm 2.9\%$ correspondingly. A sole significant (by a very low margin) overhead is $0.35\% \; \pm 0.33\%$ at 64 cores.
We measured the mean overhead for Jacobi benchmark for an input of $2^{3217}-1$ for up to 256 cores.
We reach the flattering value $-3.8\% \; \pm 0.93\%$ at 16 cores in favour of PArrows, it was the sole significant overhead value. The  value for 256 cores was $0.31\% \; \pm 0.39\%$.
Mean overhead for distributed Gentleman multiplication was also low. Significant values include $1.23\% \; \pm 1.20\%$ at 64 cores and $2.4\% \; \pm 0.97\%$ at 256 cores. It took PArrows 64.2 seconds at 256 cores to complete the benchmark.

Similar to the shared memory setting, PArrows only imply a very low penalty with distributed memory that lies in lower single-percent digits at most.

\subsection{Discussion}

%\bestAndWorstBenchmarks

\begin{table}[]
\centering
\caption{Overhead in the shared memory benchmarks. Bold marks values
  in favour of PArrows. Overhead is unit-less, multiply by 100 to
  obtain percent. Runtime is in seconds.}
\label{tab:meanOverheadSharedMemory}
\centering
\begingroup\catcode`"=9
\begin{tabular}{C{4cm} C{2cm} C{1.7cm} C{2.3cm} C{1.7cm}}
	\thead{Benchmark} & \thead{Base}             & \thead{Mean of \\ mean \\overheads} & \thead{Maximum \\ normalised \\ stdDev} & \thead{Runtime for \\ 16 cores} \\ \hline \\
	\multirow{3}{*}{\sudokutest 1000}
	\csvreader[head to column names]{benchmarks/sudoku-sm/bestAndWorstSudoku-1000.csv}{}{& \vs & \meanOverhead & \maxStdDevForOverhead & \runtimeMaxCores \\}
	\\ \hline \\
	\multirow{1}{*}{\torustest 512}
	\csvreader[head to column names]{benchmarks/torus-sm/bestAndWorstTorusSM-512.csv}{}{& \vs & \meanOverhead & \maxStdDevForOverhead & \runtimeMaxCores \\}
	\\ \hline \\
	\multirow{3}{*}{\rmtest 11213 32}
	\csvreader[head to column names]{benchmarks/sm-rm/bestAndWorstRMSM-11213-32.csv}{}{& \vs & \meanOverhead & \maxStdDevForOverhead & \runtimeMaxCores \\}
	\\ \hline \\
	\multirow{3}{*}{\rmtest 11213 64}
	\csvreader[head to column names]{benchmarks/sm-rm/bestAndWorstRMSM-11213-64.csv}{}{& \vs & \meanOverhead & \maxStdDevForOverhead & \runtimeMaxCores \\}
\end{tabular}
\endgroup
\end{table}

\begin{table}[]
\centering
\caption{Overhead in the distributed memory benchmarks. Bold marks values
  in favour of PArrows. Overhead is unit-less, multiply by 100 to
  obtain percent. Runtime is in seconds.}
\olcomment{Vielleicht gleich durchmultiplizieren und \% angeben? Nicht
zu viele leere Zeilen in der Tabelle?}
\label{tab:meanOverHeadDistributedMemory}
\centering
\begingroup\catcode`"=9
\begin{tabular}{C{4cm} C{2cm} C{1.7cm} C{2.3cm} C{1.7cm}}
	\thead{Benchmark} & \thead{Base}             & \thead{Mean of \\ mean \\overheads} & \thead{Maximum \\ normalised \\ stdDev} & \thead{Runtime for \\ 256 cores} \\ \hline \\
	\multirow{1}{*}{\torustest 4096}
	\csvreader[head to column names]{benchmarks/distributed-torus/bestAndWorstTorus-4096.csv}{}{& \vs & \meanOverhead & \maxStdDevForOverhead & \runtimeMaxCores \\}
	\\ \hline \\
  	\multirow{1}{*}{\rmtest 44497 256}
	\csvreader[head to column names]{benchmarks/distributed-rm/bestAndWorstRM-44497-256.csv}{}{& \vs & \meanOverhead & \maxStdDevForOverhead & \runtimeMaxCores \\}
	\\ \hline \\
	\multirow{1}{*}{\jacobitest 3217}
	\csvreader[head to column names]{benchmarks/distributed-jacobi/bestAndWorstJacobi-3217.csv}{}{& \vs & \meanOverhead & \maxStdDevForOverhead & \runtimeMaxCores \\}
\end{tabular}
\endgroup
\end{table}


PArrows performed in our benchmarks with little to no overhead. Tables~\ref{tab:meanOverheadSharedMemory} and \ref{tab:meanOverHeadDistributedMemory} clarify this once more: The PArrows-enabled versions trade blows with their vanilla counterparts when comparing the means over all cores of the mean overheads. If we combine these findings with the usability of our DSL,
the minor overhead induced by PArrows is outweighed by their convenience and usefulness to the user.

PArrows is an extendable formalism, they can be easily ported to further parallel Haskells while still maintaining interchangeability. It is straightforward to provide further implementations of algorithmic skeletons in PArrows.
