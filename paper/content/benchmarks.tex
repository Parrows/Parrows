\section{Performance results}
\label{sec:benchmarks}

\newcommand{\rmtest}{Rabin--Miller test\xspace}
\newcommand{\sudokutest}{Sudoku test\xspace}
\newcommand{\torustest}{Torus Matrix Multiplication test\xspace}
\newlength{\plotwidthSMP}
\setlength{\plotwidthSMP}{0.39\textwidth}
\newlength{\plotwidthDist}
\setlength{\plotwidthDist}{0.6\textwidth}


\newcommand{\performanceplot}[7]{
\begin{tikzpicture}
\begin{axis}[title={#1},
title style={align=center},
scale only axis, width=#7,
xlabel=Cores,
%xtick=data,
xtick distance=#4,
ylabel=Time (s),
ylabel near ticks,
minor tick num=2,
grid=major,
legend entries={#2},
legend style={at={(0.99,0.99)},anchor=north east},
max space between ticks=50pt,
grid style={line width=.1pt, draw=gray!10},
major grid style={line width=.2pt,draw=gray!50},
xmin=-1,
xmax=#6]
#5
\end{axis}
\end{tikzpicture}
}

\newcommand{\performancediffplot}[8]{
\begin{tikzpicture}
\begin{axis}[title={#1},
title style={align=center},
scale only axis, width=#8,
xlabel=Cores,
%xtick=data,
ytick distance=#6,
xtick distance=#4,
minor tick num=9,
ylabel=Absolute time difference (s),
ylabel near ticks,
grid=both,
legend entries={#2},
legend style={at={(0.99,0.99)},anchor=north east},
max space between ticks=50pt,
grid style={line width=.1pt, draw=gray!10},
major grid style={line width=.2pt,draw=gray!50},
xmin=-1,
xmax=#7]
#5
\end{axis}
\end{tikzpicture}
}

\newcommand{\speedupplot}[7]{
\begin{tikzpicture}
\begin{axis}[title={#1},
title style={align=center},
scale only axis, width=#7,
xlabel=Cores,
%xtick=data,
%ytick=data,
xtick distance=#4,
ytick distance=#4,
ylabel=Speedup,
ylabel near ticks,
grid=major,
legend entries={linear, #2},
legend style={at={(0.01,0.99)},anchor=north west},
max space between ticks=50pt,
grid style={line width=.1pt, draw=gray!10},
major grid style={line width=.2pt,draw=gray!50},
ymin=-1,
xmin=-1,
ymax=#6,
xmax=#6]
\addplot [domain=0:#3, no markers,dotted,thick]{x};
#5
\end{axis}
\end{tikzpicture}
}

\newcommand{\speedupdiffplot}[7]{
\begin{tikzpicture}
\begin{axis}[title={#1},
title style={align=center},
scale only axis, width=#7,
xlabel=Cores,
%xtick=data,
xtick distance=#4,
ytick distance=0.5,
ylabel=Absolute speedup difference,
ylabel near ticks,
grid=major,
legend entries={#2},
legend style={at={(0.99,0.99)},anchor=north east},
max space between ticks=50pt,
grid style={line width=.1pt, draw=gray!10},
major grid style={line width=.2pt,draw=gray!50},
xmin=-1,
xmax=#6]
#5
\end{axis}
\end{tikzpicture}
}

\newcommand{\performanceSkelRMSM}[2]{
\performanceplot{Parallel run time of \rmtest ``#2''}{Eden CP, Multicore, Par Monad}{16}{4}{
\addplot+ [very thick] table [scatter, x="nCores", y="time", col sep=comma, mark=none,
smooth]{benchmarks/sm-rm/bench-sm-rm.bench.skelrm-parr-eden-cp-#1-#2.csv};
\addplot+ [very thick] table [scatter, x="nCores", y="time", col sep=comma, mark=none,
smooth]{benchmarks/sm-rm/bench-sm-rm.bench.skelrm-parr-mult-#1-#2.csv};
\addplot+ [very thick] table [scatter, x="nCores", y="time", col sep=comma, mark=none,
smooth]{benchmarks/sm-rm/bench-sm-rm.bench.skelrm-parr-par-#1-#2.csv};
}{17}{\plotwidthSMP}
}

\newcommand{\speedupSkelRMSM}[2]{
\speedupplot{Speedup of \rmtest ``#2''}{Eden CP, Multicore, Par Monad}{16}{4}{
\addplot+ [very thick] table [scatter, x="nCores", y="speedup", col sep=comma, mark=none,
smooth]{benchmarks/sm-rm/bench-sm-rm.bench.skelrm-parr-eden-cp-#1-#2.csv};
\addplot+ [very thick] table [scatter, x="nCores", y="speedup", col sep=comma, mark=none,
smooth]{benchmarks/sm-rm/bench-sm-rm.bench.skelrm-parr-mult-#1-#2.csv};
\addplot+ [very thick] table [scatter, x="nCores", y="speedup", col sep=comma, mark=none,
smooth]{benchmarks/sm-rm/bench-sm-rm.bench.skelrm-parr-par-#1-#2.csv};
}{17}{\plotwidthSMP}
}

%\performanceSkelRMSM{4423}{32}

%\speedupSkelRMSM{4423}{32}

%\performanceSkelRMSM{4423}{64}

%\speedupSkelRMSM{4423}{64}



%\performanceSkelRMSM{9941}{32}

%\speedupSkelRMSM{9941}{32}{32}

%\performanceSkelRMSM{9941}{64}

%\speedupSkelRMSM{9941}{64}



%\performanceSkelRMSM{11213}{32}


\begin{figure}
%\centering
%\performanceSkelRMSM{11213}{64}\hfill%
{\speedupSkelRMSM{11213}{32}}\hfill%
{\speedupSkelRMSM{11213}{64}}
\caption{Relative speedup of \rmtest on a multicore machine. We used the same PArrows-based implementation with
  different backends on the same hardware. Measurements were performed on a single node of the Glasgow
  grid; it has 16 real cores. Input was $2^{11213}-1$, we used 32 (left) or 64 (right)
  tasks. The
  closer to linear speedup the better.}
\label{fig:bench-rm-sm}
\end{figure}

% \newcommand{\performanceSkelRMDist}[4]{
% \performanceplot{Parallel run time \\ SkelRM #1 #2}{PArrows}{256}{#3}{
% \addplot table [scatter, x="nCores", y="time", col sep=comma, mark=none,
% smooth]{benchmarks/distributed-rm/bench-distributed.bench.skelrm-parrows-#1-#2.csv};
% % \addplot table [scatter, x="nCores", y="time", col sep=comma, mark=none,
% % smooth]{benchmarks/distributed-rm/bench-distributed.bench.skelrm-eden-#1-#2.csv};
% }{#4}
% }

\newcommand{\speedupSkelRMDist}[4]{
\speedupplot{Speedup of \rmtest ``#1,#2''}{PArrows}{256}{#3}{
% \addplot [mark=*,very thick] table [scatter, x="nCores", y="speedup", col sep=comma, mark=none,
% smooth]{benchmarks/distributed-rm/bench-distributed.bench.skelrm-parrows-11213-#2.csv};
\addplot [mark=*,very thick,blue] table [scatter, x="nCores", y="speedup", col sep=comma, mark=none,
smooth]{benchmarks/distributed-rm/bench-distributed.bench.skelrm-parrows-#1-#2.csv};
% \addplot table [scatter, x="nCores", y="speedup", col sep=comma, mark=none,
% smooth]{benchmarks/distributed-rm/bench-distributed.bench.skelrm-eden-#1-#2.csv};
}{#4}{\plotwidthDist}
}

\newcommand{\performanceSkelRMDistDiff}[5]{
\performancediffplot{Run time differences\\for \rmtest ``#1, #2''}{(Eden $-$ PArrows)}{256}{#3}{
\addplot+[mark=*,very thick,error bars/.cd,
    y dir=both,y explicit] table [x="nCores", y="time", y error="max stddev", col sep=comma, mark=dots,
smooth]{benchmarks/distributed-rm/#1-#2-diff.csv};
}{#4}{#5}{\plotwidthDist}
}


% \newcommand{\speedupSkelRMDistDiff}[4]{
% \speedupdiffplot{Parallel speedup difference \\ of SkelRM #1 #2}{(Eden - PArrows)}{512}{#3}{
% \addplot table [scatter, x="nCores", y="speedup", col sep=comma, mark=none,
% smooth]{benchmarks/distributed-rm/#1-#2-diff.csv};
% }{#4}
% }

%\performanceSkelRMDist{9941}{256}{16,32,64,128,256,512}{528}

%\performanceSkelRMDistDiff{9941}{256}{16,32,64,128,256,512}{0.11}{528}

%\speedupSkelRMDist{9941}{256}{16,32,64,128,256,512}{528}

%\speedupSkelRMDistDiff{9941}{256}{16,32,64,128,256,512}{528}



%\performanceSkelRMDist{11213}{256}{16,32,64,128,256,512}{528}

%\performanceSkelRMDistDiff{11213}{256}{16,32,64,128,256,512}{0.1}{528}

%\speedupSkelRMDist{11213}{256}{16,32,64,128,256,512}{528}

%\speedupSkelRMDistDiff{11213}{256}{16,32,64,128,256,512}{528}


\begin{figure}
\centering
%\performanceSkelRMDist{44497}{256}{32,64,128,256,512}{544}
%
{\speedupSkelRMDist{44497}{256}{32}{260}\label{subfig:rm-dist-a}}%
%\hfill%
{\performanceSkelRMDistDiff{44497}{256}{32}{0.5}{260}\label{subfig:rm-dist-b}}
\caption{Parallel performance of \rmtest on the Glasgow grid
  consisting of 256 cores. Input was $2^{44497}-1$, we used 256
  tasks. The top plot shows absolute speedup in a distributed memory setting. The
  closer to linear speedup the better. Time
  (and hence speedup) measurements for PArrows with Eden backend and
  Eden almost coincide. Hence, bottom plot shows
absolute time differences for this benchmark. The
lower the value, the better for PArrows\olcomment{CHECKME}.}
\label{fig:bench-rm-dist}
\end{figure}

\newcommand{\speedupTorusDist}[3]{
\speedupplot{Speedup of \torustest ``#1''}{PArrows}{256}{#2}{
\addplot [mark=*,very thick,blue] table [scatter, x="nCores", y="speedup", col sep=comma, mark=none,
smooth]{benchmarks/distributed-torus/bench-torus-distributed.bench.torus-matrix-parrows-#1.csv};
}{#3}{\plotwidthDist}
}

\newcommand{\performanceTorusDistDiff}[4]{
\performancediffplot{Run time differences\\for \torustest ``#1''}{(Eden $-$ PArrows)}{256}{#2}{
\addplot+[mark=*,very thick,error bars/.cd,
    y dir=both,y explicit] table [x="nCores", y="time", y error="max stddev", col sep=comma, mark=dots,
smooth]{benchmarks/distributed-torus/#1-diff.csv};
}{#3}{#4}{\plotwidthDist}
}

\begin{figure}
\centering
%\performanceSkelRMDist{44497}{256}{32,64,128,256,512}{544}
%
{\speedupTorusDist{1024}{32}{260}\label{123123123}}%
%\hfill%
{\performanceTorusDistDiff{1024}{32}{0.5}{260}\label{12312312312}}
\caption{Parallel performance of \torustest on the Glasgow grid
  consisting of 256 cores. Input was a matrix size of $1024$. The top plot shows absolute speedup in a distributed memory setting. The
  closer to linear speedup the better. Time
  (and hence speedup) measurements for PArrows with Eden backend and
  Eden almost coincide. Hence, bottom plot shows
absolute time differences for this benchmark. The
lower the value, the better for PArrows\olcomment{CHECKME}.}
\label{123123123122}
\end{figure}

\newcommand{\performanceSudokuSM}[1]{
\performanceplot{Parallel run time of \sudokutest ``#1''}{Eden CP, Multicore, Par Monad, Par Monad (vanilla)}{16}{4}{
\addplot+ [very thick] table [scatter, x="nCores", y="time", col sep=comma, mark=none,
smooth]{benchmarks/sudoku-sm/bench-sudoku-sm.bench.parrows-sudoku-parmap-eden-sudoku17.#1.txt.csv};
\addplot+ [very thick] table [scatter, x="nCores", y="time", col sep=comma, mark=none,
smooth]{benchmarks/sudoku-sm/bench-sudoku-sm.bench.parrows-sudoku-parmap-mult-sudoku17.#1.txt.csv};
\addplot+ [very thick] table [scatter, x="nCores", y="time", col sep=comma, mark=none,
smooth]{benchmarks/sudoku-sm/bench-sudoku-sm.bench.parrows-sudoku-parmap-par-sudoku17.#1.txt.csv};
\addplot+ [very thick] table [scatter, x="nCores", y="time", col sep=comma, mark=none,
smooth]{benchmarks/sudoku-sm/bench-sudoku-sm.bench.parmonad-sudoku-sudoku17.#1.txt.csv};
}{17}{\plotwidthSMP}
}

\newcommand{\speedupSudokuSM}[1]{
\speedupplot{Speedup of \sudokutest ``#1''}{Eden CP, Multicore, Par Monad, Par Monad (vanilla)}{16}{4}{
\addplot+ [very thick] table [scatter, x="nCores", y="speedup", col sep=comma, mark=none,
smooth]{benchmarks/sudoku-sm/bench-sudoku-sm.bench.parrows-sudoku-parmap-eden-sudoku17.#1.txt.csv};
\addplot+ [very thick] table [scatter, x="nCores", y="speedup", col sep=comma, mark=none,
smooth]{benchmarks/sudoku-sm/bench-sudoku-sm.bench.parrows-sudoku-parmap-mult-sudoku17.#1.txt.csv};
\addplot+ [very thick] table [scatter, x="nCores", y="speedup", col sep=comma, mark=none,
smooth]{benchmarks/sudoku-sm/bench-sudoku-sm.bench.parrows-sudoku-parmap-par-sudoku17.#1.txt.csv};
\addplot+ [very thick] table [scatter, x="nCores", y="speedup", col sep=comma, mark=none,
smooth]{benchmarks/sudoku-sm/bench-sudoku-sm.bench.parmonad-sudoku-sudoku17.#1.txt.csv};
}{17}{\plotwidthSMP}
}

\begin{figure}
\centering
%\performanceSkelRMDist{44497}{256}{32,64,128,256,512}{544}
%
{\speedupSudokuSM{1000}\label{1231231232}}%
%\hfill%
{\performanceSudokuSM{1000}\label{1231231123122}}
\caption{Relative speedup of \rmtest on a multicore machine. We used the same PArrows-based implementation with
  different backends on the same hardware and the |parMap| version from the |Par| Monad examples \citet{par-monad}\mbcomment{\\originally from http://community.haskell.org/~simonmar/par-tutorial.pdf}. Measurements were performed on a single node of the Glasgow
  grid; it has 16 real cores. Input was a file of $1000$ Sudokus. The
  closer to linear speedup the better.}
\label{1231231233122}
\end{figure}



%\speedupSkelRMDistDiff{44497}{256}{32,64,128,256,512}{544}

%% %%%%%%%%%%%%%%%%%%%%%%%%%%%%%%%%%%%%%%%%%%%%%%%%%%%%%%%%%%%%
%% Text

\subsection{Hardware}

We have tested our parallel DSL and algorithmic skeletons implemented
in it. Benchmarks were conducted both in a shared and in a distributed
memory setting. The hardware was the ``Glasgow grid'', consisting of
16 machines with 16 cores. \olcomment{insert typical blabla: consisting of
  Intel XXXYYYY at XX~GHz with XXXX~GB RAM and FOOBAR Linux version
  XXXX with kernel XXXXXXXX.}
Each core had a hyperthreading capability,
so the total number of reported processors per machine was
32. Non-surprisingly, we found that hyperthreaded 32 cores do not
behave in the same manner as real 16 cores (numbers here for a single
machine). We disregarded the hyperthreading ability in most of the
cases.

We used a single machine with 16 real cores as a shared memory testbed
and the whole grid with 256 real cores as a device to test our
distributed memory software.

\olcomment{only if we do it: list also our i7-6770K machine}

\subsection{Test programs}

We used multiple tests that originated from different
sources. Most of them are parallel mathematical computations, initially
implemented in Eden. Table~\ref{tab:benches} summarizes.

\begin{table}
\caption{The benchmarks we use in this paper.}
\label{tab:benches}
\centering
%% something was wrong with separators in table
\renewcommand{\tabcolsep}{1em}
\begin{tabular}{lccll}
\toprule
Name & Area & Type & Origin & Citation \\
\midrule
\rmtest & Mathematics & |parMap+reduce| & Eden & \citet{Lobachev2012}\\
Jacobi sum test & Mathematics & |workpool+reduce| & Eden & \citet{Lobachev2012}\\
Gentleman & Mathematics & |torus| & Eden & \citet{Loogen2012}\\
Sudoku & Puzzle & |parMap| & |Par| Monad & \citet{par-monad} \mbcomment{originally from: http://community.haskell.org/~simonmar/par-tutorial.pdf}\\
\bottomrule
\end{tabular}
\end{table}

\rmtest is a probabilistic primality test that iterates multiple (32--256 here)
``subtests''. Should a subtest fail, the input is definitely not a
prime. If all $n$ subtest pass, the input is composite with the
probability of $1/4^{n}$. 

Jacobi sum test or APRCL is also a primality test, that however,
guarantees the correctness of the result. It is probabilistic in the
sense that its run time is not certain. Unlike \rmtest, the subtests
of Jacobi sum test have very different durations. \citet{lobachev-phd}
\olcomment{discuss some optimisations of parallel APRCL.} Generic parallel
implementation of \rmtest and APRCL were presented in \citet{Lobachev2012}.

``Gentleman'' is a standard Eden test program, developed
for their |torus| skeleton. It implements a parallel matrix
multiplication \citep{Gentleman1978}. We ported its latest Eden
version \citep{Loogen2012} to PArrows.

A~parallel Sudoku solver was used by \citet{par-monad} to compare |Par| Monad
to Multicore Haskell. We ported it to PArrows.



\subsection{What parallel Haskells run where}

The \ensuremath{\Conid{Par}} monad and Multicore Haskell can be executed on a shared
memory machines only. Although GpH is available on distributed memory
clusters, and newer distributed memory Haskells such as HdpH exist,
current support of distributed memory in PArrows is limited to
Eden. We used the MPI backend of Eden in a distributed memory
setting. However, for shared memory Eden features a ``CP'' backend
that merely copies the memory blocks between distributed heaps. In
this mode Eden still operates in the ``nothing shared'' setting, but
is adapted better to multicore machines. We label this version of Eden
in the plots as ``Eden~CP''.



\subsection{Effect of hyperthreading}

The PArrows version of \rmtest on a single node of the Glasgow grid
showed almost linear speedup (Fig.~\ref{fig:bench-rm-sm}). The speedup
of 64-task PArrows/Eden at 16 real cores version was 13.65, the
efficiency was 85.3\%.  However, if we increase the number of
requested cores to be 32---\ie if we use hyperthreading on 16 real
cores---the speedup does not increase that well. It is merely 15.99
for 32 tasks with PArrows/Eden. It is worse for other backends.  As
for 64 tasks, we obtain the speedup of 16.12 with PArrows/Eden at 32
hyperthreaded cores and only 13.55 with PArrows/Multicore
Haskell. Efficiency is 50.4\% and 42.3\%, respectively. The Eden
version used here was Eden~CP, the ``share nothing'' SMP build.

In the distributed memory setting the same effect ensues. We obtain
plummeting speedup of 124.31 at 512 hyperthreaded cores, whereas it was
213.172 for 256 real cores. Apparently, hyperthreading in the Glasgow
grid fails to execute two parallel Haskell processes with full-fledged
parallelism. For this reason, we did not regard hyperthreaded cores in
our speedup plots in Figs.~\ref{fig:bench-rm-sm}--XXXXXXXX.


% rm 11213 32 32-sm speedup eden: 15.993037587283924
% -"- multi: 15.09948017762912
% -"- par: 14.909092857846693

% -"- 64 32-sm speedup eden: 16.118040224478424
% -"- multi: 13.545304115702333
% -"- par: 15.155709987503396

\subsection{Benchmark results}

The difference between, say, PArrows with |Par| Monad backend and a
genuine |Par|
Monad benchmark is very small. To give an example, it is XXXXXXXX for
XXXXXXXXX and XXXXXXXXXX in the shared memory setting. It is almost invisible in speedup and
(non shown) run time plots. We thus show only the results for the
PArrows-enabled versions.

To showcase that the induced overhead of
PArrows is small, we plot execution time differences between measurements for
PArrows and the corresponding backend in a separate plot
(Figs.~\ref{fig:bench-rm-dist}, XXXXXXX). The differences range in
about 0.5~seconds for the execution time of 46~seconds on 256 cores
for distributed \rmtest with PArrows and Eden. The plots show absolute
time differences that are not relativated \wrt the total execution time.

The error bars ends were computed from pointwise maximum of both standard
deviations from both measurements for PArrows and non-PArrows
versions. These are the values provided by the |bench| package that we
used for benchmarking. We call a difference between two versions
significant when the border of the error bar of absolute time
difference is above or below zero. In other words: the time
difference is significant if it is above measurement error.

\subsubsection{\rmtest}
\olcomment{THE ACTUAL TEXT IS MISSING. What do we see in the plots?
  Why is it good?}
The multicore version of our parallel \rmtest benchmark is in
Figure~\ref{fig:bench-rm-sm}. We have executed the test with 32 and 64
tasks. The plot shows the PArrows-enabled versions with corresponding backends.
The performance of PArrows/Eden~CP in shared memory is slightly better than
for SMP variants such as PArrows/Multicore Haskell and PArrows/|Par|
Monad. The reason for this behavior lies in the backend in our reasoning. One
possible explanation is garbage collection. It is easier to GC
(independently) $n$ heaps than to GC one large. This agrees with known
results\olcomment{WHICH EXACTLY?} that motivate the efforts for
parallel GC in SMP-based Haskells.

To show that PArrows induce very small overhead, we compare the ported
versions of the benchmark to the PArrows-enabled ones with
corresponding backend.\olcomment{THE RESULT?}

In the distributed memory setting, we utilize PArrows/Eden. Eden in
its turn used the MPI backend. We see an almost linear speedup of
\rmtest with 256 tasks and input $2^444497-1$. The sequential run time
was computed as mean of three consecutive executions on a single
core---the single run took two hours 43 minutes. The zero difference  between
PArrows/Eden and Eden almost always lies on the error bar of
the measurement. The only exceptions, where PArrows version was slower
and it was significant, were for 64 and
512 cores, for 0.49 and 0.26~second, respectively. This corresponds to
0.30\% and 0.33\% relative time difference, respectively. The 512 core
version suffers, of course, from the aforementioned hyperthreading
problem. The PArrows-induced overhead was merely 0.56\% for 256
cores. The PArrows version was 0.30\% \emph{faster} for 128 cores.
