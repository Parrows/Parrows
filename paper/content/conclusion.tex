
\section{Conclusion}
\label{sec:conclusion}
Arrows are a generic concept that allows for powerful composition
combinators. To our knowledge we are first to represent
\emph{parallel} computation with Arrows.
% \olcomment{that strange arrows-based robot interaction paper from 1993 or so! clearly discuss in related work} \done
%
Arrows turn out to be a useful tool for composing parallel
programs. We have shown, that, in order to have a generic and extensible parallel Haskell, we do not have to restrict ourselves to a monadic interface. Instead, we use Arrows which are a more general concept than Monads, but still allow for powerful composition. Furthermore, we think Arrows are a better fit to parallelize pure code than a monadic solution as regular functions are already Arrows and can be used with our DSL in a more natural way. We manage to retain this nice property of parallel Haskells such as Eden or GpH (which use pure constructs) while still having a generic and composable interface.
%
The DSL natively allows for parallelization of monadic code via the Kleisli type and additionally allows to parallelize any Arrow type that has an instance for |ArrowChoice| (note that some skeletons require an additional |ArrowLoop| instance).
%
Parallel Arrows, as presented here, feature an implementation of the |ArrowParallel| type class for GpH Haskell, |Par| Monad, and Eden. With our approach parallel programs can be ported across these flavours with little to no effort. It is quite straightforward to add further backends.
%
%
Performance-wise, Parallel Arrows are on par with existing parallel Haskells, as they only introduce minor overhead in some of our benchmarks.
%
The benefit is, however, the greatly increased portability of parallel programs.

%\mbcomment{mention ArrowLoop in Torus and Ring chapters}
%\olcomment{Parrows + accelerate = love?} \olcomment{Metion port to Frege.}


\subsection{Future Work}
\label{sec:future-work}

Our PArrows DSL can be expanded to further parallel Haskells. More specifically we target HdpH \cite{Maier:2014:HDS:2775050.2633363} for this future extension. HdpH is a modern distributed Haskell that would benefit from our Arrow notation. Further Future-aware versions of Arrow combinators can be defined. Existing combinators could also be improved.
Arrow-based notation might enable further compiler optimizations.

More experiences with seamless porting of parallel PArrows-based programs across the backends are welcome.
Of course, we are ourselves working on expanding both our skeleton library and the number of skeleton-based parallel programs that use our DSL to be portable across flavours of parallel Haskells.
It would also be interesting to see a hybrid of PArrows and Accelerate \cite{McDonell:2015:TRC:2887747.2804313}.
Ports of our approach to other languages such as Frege, Eta, or Java directly are in an early development stage.
