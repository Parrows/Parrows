
\section{Conclusion}
\label{sec:conclusion}
Arrows are a generic concept that allows for powerful composition
combinators. To our knowledge we are first to represent
\emph{parallel} computation with
Arrows, and hence to show their usefulness for
% Arrows are a useful tool for
composing parallel
programs. We have shown that for a generic and extensible parallel Haskell, we do not have to restrict ourselves to a monadic interface. % Instead, we use Arrows which are a more general concept than Monads, but still allow for powerful composition. Furthermore, we think
We argue that Arrows are better suited to  parallelise
pure functions than Monads,  as the functions are already Arrows and can be used
directly in our DSL.
Arrows are a better fit to parallelise pure code than a monadic solution as regular functions are already Arrows and can be used with our DSL in a more natural way. % We manage to retain this nice property of parallel Haskells such as Eden or GpH (which use pure constructs)
% while still having a generic and composable interface.
We use a non-monadic interface (similar to Eden or GpH) and retain composibility.
%
The DSL allows for a direct parallelisation of monadic code via the Kleisli type and additionally allows to parallelise any Arrow type that has an instance for |ArrowChoice|. (Some skeletons require an additional |ArrowLoop| instance.)


We have demonstrated the generality of the approach by exhibiting PArrow implementations for Eden, GpH, and the |Par| Monad. Hence, parallel programs can be ported between task parallel Haskell implementations with little or no effort. We are confident that it will be straightforward to add other task-parallel Haskells.
% Parallel Arrows, as presented here, feature an implementation of the |ArrowParallel| type class for GpH Haskell, |Par| Monad, and Eden. With our approach parallel programs can be ported across these flavours with little to no effort. It is quite straightforward to add further backends.
% %
% %
% Performance-wise,
%Parallel Arrows are on par with existing parallel Haskells, as they only introduce minor overhead in some of our benchmarks.
%
%The benefit is, however, the
In other words, PArrows greatly increase portability of parallel Haskell programs.
%
%
Our measurements of four benchmarks on both shared and distributed memory platforms shows that the generality and portability of PArrows has very low performance overheads, \ie never more than $8\% \; \pm 6.9\%$ and typically under $2\%$.

%\mbcomment{mention ArrowLoop in Torus and Ring chapters}
%\olcomment{Parrows + accelerate = love?} \olcomment{Metion port to Frege.}


\subsection{Future work}
\label{sec:future-work}

Our PArrows DSL can be expanded % to further parallel Haskells.  More specifically we target HdpH \cite{Maier:2014:HDS:2775050.2633363} for this future extension. HdpH is a modern distributed Haskell that would benefit from our Arrow notation.
to other task parallel Haskells, and a specific target is HdpH \cite{Maier:2014:HDS:2775050.2633363}.
Further Future-aware versions of Arrow combinators can be defined. Existing combinators could also be improved, for example a more special versions of |>>>| and |***| combinators are viable.
%Arrow-based notation might enable further compiler optimisations, such as 

% More experiences with seamless porting of parallel PArrows-based programs across the backends are welcome.
% Of course, we are ourselves working on expanding
In ongoing work we are expanding 
both our skeleton library and the number of skeleton-based parallel programs that use our DSL.
% to be portable across flavours of parallel Haskells.
It would also be interesting to see a hybrid of PArrows and Accelerate \cite{McDonell:2015:TRC:2887747.2804313}.
Ports of our approach to other languages such as Frege, Eta, or Java directly are at an early development stage.
