
\section{Conclusion}
\label{sec:conclusion}
Arrows are a generic concept that allows for powerful composition combinators. To our knowledge we are the first ones to represent parallel computation with arrows.\olcomment{that strange arrows-based robot interaction paper from 1993 or so! clearly discuss in related work} \done

Arrows turn out to be a useful tool for composing in parallel programs. We do not have to introduce new monadic types that wrap the computation. Instead use arrows just like regular sequential pure functions. 
%
This work features multiple parallel backends: the already available parallel Haskell flavours. Parallel Arrows feature an implementation of the |ArrowParallel| interface for Multicore Haskell, |Par| Monad, and Eden. With our approach parallel programs can be ported across these flavours with little to no effort.
%
%
Performance-wise, Parallel Arrows are on par with existing parallel Haskells, as they do not introduce any notable overhead.

The strictness problems of the |Par| Monad backend materializes with |ArrowLoop|. Additional work is required to ensure correct behaviour of the |Par| Monad in this context.

\mbcomment{mention ArrowLoop in Torus and Ring chapters}
\olcomment{Parrows + accelerate = love? Metion port to Frege.}


\subsection{Future Work}
\label{sec:future-work}

Our PArrows DSL can be expanded to futher parallel Haskells. More specifically we target HdpH \cite{Maier:2014:HDS:2775050.2633363}, a modern distributed Haskell that would benefit from our Arrows notation. More Future-aware versions of Arrow combinators can be defined and existing can be further improved. We would look into more transparency of the DSL, it should basically infuse as little overhead as possible.

We are looking into more experiences with seamless porting of parallel PArrow-based programs across the backends.
Of course, we are working on expanding both our skeleton library and the number of skeleton-based parallel programs that use our DSL to be portable across flavours of parallel Haskells.
It would also be interesting to see a hybrid of PArrows and Accelerate.
Ports of our approach to other languages like Frege or Java directly are in an early development stage.
