
\section{Conclusion}
\label{sec:conclusion}
Arrows are a generic concept that allows for powerful composition
combinators. To our knowledge we are first to represent
\emph{parallel} computation with Arrows.
% \olcomment{that strange arrows-based robot interaction paper from 1993 or so! clearly discuss in related work} \done
%
Arrows turn out to be a useful tool for composing parallel
programs. We do not have to introduce new monadic types that wrap the
computation. Instead, we use Arrows in the same manner one uses sequential pure functions. 
%
This work features multiple parallel backends: the already available parallel Haskell flavours. Parallel Arrows, as presented here, feature an implementation of the |ArrowParallel| type class for GpH Haskell, |Par| Monad, and Eden. With our approach parallel programs can be ported across these flavours with little to no effort. It is quite straightforward to add further backends. 
%
%
Performance-wise, Parallel Arrows are on par with existing parallel Haskells, as they only introduce minor overhead in some of our benchmarks.
%
The benefit is, however, the greatly increased portability of parallel programs.

%\mbcomment{mention ArrowLoop in Torus and Ring chapters}
%\olcomment{Parrows + accelerate = love?} \olcomment{Metion port to Frege.}


\subsection{Future Work}
\label{sec:future-work}

Our PArrows DSL can be expanded to further parallel Haskells. More specifically we target HdpH \cite{Maier:2014:HDS:2775050.2633363} for this future extension. HdpH is a modern distributed Haskell that would benefit from our Arrows notation. Further Future-aware versions of Arrow combinators can be defined. Existing combinators could also be improved.
Arrow-based notation might enable further compiler optimizations.

More experiences with seamless porting of parallel PArrows-based programs across the backends are welcome.
Of course, we are working ourselves on expanding both our skeleton library and the number of skeleton-based parallel programs that use our DSL to be portable across flavours of parallel Haskells.
It would also be interesting to see a hybrid of PArrows and Accelerate \cite{McDonell:2015:TRC:2887747.2804313}.
Ports of our approach to other languages like Frege or Java directly are in an early development stage.
