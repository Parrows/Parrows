\section{Arrows}

John Hughes defined the Arrow typeclass as follows\citHughes:
\begin{lstlisting}[frame=htrbl]
class Arrow a where
	arr :: (b -> c) -> a b c
	(>>>) :: a b c -> a c d -> a b d
	first :: a b c -> a (b,d) (c,d)
\end{lstlisting}
\lstinline{arr :: (b -> c) -> a b c} is used to lift an ordinary function to an Arrow type. This can be thought of as analogous to the monadic \lstinline{return}. The \lstinline{>>>} operator, in a similar way, is analogous to the monadic composition operator \lstinline{>>=}. Ant lastly, the \lstinline{first} operator, which takes the input arrow from b to c and converts it into an arrow on pairs with the second argument untouched, is also needed for actual useful code as without it, we wouldn't have a way to save input across arrows.
\\\\
The most prominent instances of this interface are regular functions, \lstinline{(->)}
\begin{lstlisting}[frame=htrbl]
instance Arrow (->) where
	arr f = f
	f >>> g = g . f
	first f = \(b, d) -> (f b, d) 
\end{lstlisting}
and the Kleisli type.
\begin{lstlisting}[frame=htrbl]
instance Arrow (Kleisli m) where
	arr f = Kleisli $ return . f
	f >>> g = Kleisli $ \b -> f b >>= g
	first f = Kleisli $ \(b,d) -> f b >>= \c -> return (c,d)
\end{lstlisting}
With this typeclass in place, Hughes also defined some syntactic sugar: The mirrored version of \lstinline{first}, called \lstinline{second},
\begin{lstlisting}[frame=htrbl]
second :: Arrow a => a b c -> a (d, b) (d, c)
second f = arr swap >>> first f >>> arr swap
	where swap (x, y) = (y, x)
\end{lstlisting}
the *** combinator which combines first and second to handle two inputs in one arrow,
\begin{lstlisting}[frame=htrbl]
(***) :: Arrow a => a b c -> a d e -> a (b, d) (c, e)
f *** g = first f >>> second g
\end{lstlisting}
and the \&\&\& combinator that constructs an arrow which outputs 2 different values like ***, but takes only one input.
\begin{lstlisting}[frame=htrbl]
(&&&) :: Arrow a => a b c -> a b d -> a b (c, d)
f &&& g = arr (\b -> (b, b)) >>> (f *** g)
\end{lstlisting}
A short example given by Hughes on how to use this is \lstinline{add} over arrows:
\begin{lstlisting}[frame=htrbl]
add :: Arrow a => a b Int -> a b Int -> a b Int
add f g = (f &&& g) >>> arr (\(u, v) -> u + v)
\end{lstlisting}
The benefit of this interface is now that any type which is shown to be an Arrow can no be used in conjunction with this newly created \lstinline{add} combinator. Even though this example is quite simple, the power of the \lstinline{Arrow} interface immediately is clear: If a type is an \lstinline{Arrow}, it can immediately used together with every library that works on \lstinline{Arrows}. With simple \lstinline{Monads} this type of extensibility is not possible.
\\\\
\textit{Note: In the definitions above we used the notation \code{a b c} for an arrow from \code{b} to \code{c}. From now on we will use the equivalent defintion \code{arr a b} for an arrow from \code{a} to \code{b} to make it easier to find the arrow type in type signatures. We kept the original notation \code{a b c} for this section to not change too much from Hughes' original definitions.}