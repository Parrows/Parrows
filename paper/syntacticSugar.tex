\subsection{Syntactic Sugar} \label{syntacticSugar}
For basic arrows, we have the \code{(***)} combinator (Fig.~\ref{fig:***Img},~\ref{fig:***}) which allows us to combine two arrows \code{arr a b} and \code{arr c d} into an arrow \code{arr (a, c) (b, d)} which does both computations at once. This can easily be translated into a parallel version \code{(|***|)} (Fig.~\ref{fig:|***|}) with the use of \code{parEval2}, but for this we require a backend which has an implementation that does not require any configuration (hence the \code{()} as the conf parameter in Fig.~\ref{fig:|***|}).
\begin{figure}[h]
\begin{lstlisting}[frame=htrbl]
(|***|) :: (ArrowChoice arr, ArrowParallel arr (Either a c) (Either b d) ())) =>
	arr a b -> arr c d -> arr (a, c) (b, d)
(|***|) = parEval2 ()
\end{lstlisting}
\caption{Definition of (|***|) - the parallel version of (***)}
\label{fig:|***|}
\end{figure}
% With this we can analogously to the serial \code{&&&}
We define the parallel \code{(|&&&|)} (Fig.~\ref{fig:|&&&|}) in a similar manner to its sequential pendant \code{(&&&)} (Fig.~\ref{fig:&&&Img},~\ref{fig:&&&}).
\begin{figure}[h]
\begin{lstlisting}[frame=htrbl]
(|&&&|) :: (ArrowChoice arr, ArrowParallel arr (Either a a) (Either b c) ()) =>
	arr a b -> arr a c -> arr a (b, c)
(|&&&|) f g = (arr $ \a -> (a, a)) >>> f |***| g
\end{lstlisting} % $ %% formatting
\caption{Definition of (|\&\&\&|) - the parallel version of (\&\&\&)}
\label{fig:|&&&|}
\end{figure}