\usepackage{tikz}

\usepackage{filecontents}
\usepackage{pgfplots, pgfplotstable}
\usepgfplotslibrary{statistics}

\usepackage[T1]{fontenc}
\usepackage[utf8]{inputenc}

%\usepackage{color}
\usepackage{hyperref}

\usepackage{subcaption}

\usepackage[round]{natbib}

\usepackage{csquotes}

% für Listings
\usepackage{listings}
\lstset{numbers=left, numberstyle=\tiny, numbersep=5pt, keywordstyle=\color{black}\bfseries, stringstyle=\ttfamily,showstringspaces=false,basicstyle=\footnotesize,captionpos=b}
\lstset{
	frame=none,
	xleftmargin=2pt,
	stepnumber=1,
	numbers=left,
	numbersep=5pt,
	numberstyle=\ttfamily\tiny\color[gray]{0.3},
	belowcaptionskip=\bigskipamount,
	captionpos=b,
	escapeinside={*'}{'*},
	language=haskell,
	tabsize=2,
	emphstyle={\bf},
	commentstyle=\it,
	stringstyle=\mdseries\rmfamily,
	showspaces=false,
	keywordstyle=\bfseries\rmfamily,
	columns=flexible,
	basicstyle=\small\sffamily,
	showstringspaces=false,
	morecomment=[l]\%,
}

\renewcommand{\cite}[1]{\citep{#1}}

\newcommand{\citHughes}{\citep{HughesArrows}}

\newcommand{\code}[1]{\lstinline{#1}}

%\newcommand{\fixme}[1]{\colorbox{red}{#1}}

%%%%%% more fun and typography

\usepackage{xifthen}

\newboolean{anonymous}
\setboolean{anonymous}{False}

\usepackage[british]{babel}

\usepackage{siunitx}

\usepackage{xspace}
\usepackage{xcolor}
\newcommand{\hl}[1]{\textcolor{red}{#1}}
%\newcommand{\comm}[2]{\hl{\(\spadesuit\){\bfseries #1: }}{\sffamily #2}\hl{\(\spadesuit\)}}
\newcommand{\comm}[2]{}
\newcommand{\olcomment}[1]{\comm{OL}{#1}}
\newcommand{\mbcomment}[1]{\comm{MB}{#1}}
\newcommand{\ptcomment}[1]{\comm{Phil}{#1}}
%\newcommand{\done}{\xspace\hl{done!}\xspace}
\newcommand{\fixme}{\mbcomment}

\DeclareRobustCommand{\xth}{\textsuperscript{th}\xspace}
\DeclareRobustCommand{\st}{\textsuperscript{st}\xspace}
\DeclareRobustCommand{\nd}{\textsuperscript{nd}\xspace}
\DeclareRobustCommand{\xrd}{\textsuperscript{rd}\xspace}

%% kuerzel
%\newcommand{\todo}{\textcolor{red}{\bfseries TODO!}\xspace}
\newcommand{\done}{\textcolor{green}{\bfseries done!}\xspace}
\DeclareRobustCommand{\hairspn}{\hspace{1pt}\nolinebreak}% hair space with no break

% \DeclareRobustCommand{\ie}{{i.\hairspn{}e.\nopagebreak[4] }}
% \DeclareRobustCommand{\eg}{{e.\hairspn{}g.\nopagebreak[4] }}
% \DeclareRobustCommand{\fe}{{f.\hairspn{}e.,}\nopagebreak[4] }
\DeclareRobustCommand{\ie}{{i.\hairspn{}e.~}}
\DeclareRobustCommand{\eg}{{e.\hairspn{}g.~}}
\DeclareRobustCommand{\fe}{{f.\hairspn{}e.,~}}
\DeclareRobustCommand{\ad}{{A.\hairspn{}D.\xspace}}
\DeclareRobustCommand{\bcc}{{A.\hairspn{}C.\xspace}}
\DeclareRobustCommand{\wrt}{w.\hairspn{}r.\hairspn{}t.~}
\DeclareRobustCommand{\etc}{etc.\ }
\DeclareRobustCommand{\cf}{\textit{cf.~}}
\DeclareRobustCommand{\viz}{\textit{viz.~}}
%\DeclareRobustCommand{\hof}{higher-or\-der function\xspace}

% %% the numberings
\DeclareRobustCommand{\xth}{\textsuperscript{th}\xspace}
\DeclareRobustCommand{\st}{\textsuperscript{st}\xspace}
\DeclareRobustCommand{\nd}{\textsuperscript{nd}\xspace}
\DeclareRobustCommand{\xrd}{\textsuperscript{rd}\xspace}

\newcommand{\tabsepbakup}{\tabcolsep}
%% for narrower tables

%% nice tables
\usepackage{booktabs}
%\usepackage{multirow}

%\usepackage{ifpdf} %% broken?!!
%\ifpdf
%\usepackage[activate={true,nocompatibility},kerning=true,spacing=true,tracking=true]{microtype}
%\else
%% non-pdf latex can less, let microtype itself decide
\usepackage{microtype}
%\fi
\microtypecontext{spacing=nonfrench} %% log said so

%%% JFP requirements:
%%% Harvard citing style, "(Curry 1933)".
%%% code: identifies italic, keywords bold
%%% figures: eps(!) or ps (can convert pdf), eventually provide a greyscale version, color NOT in cmyk
