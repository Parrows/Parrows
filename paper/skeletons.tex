\section{Skeletons}
With the \code{ArrowParallel} typeclass in place and implemented, we can now implement some basic parallel skeletons.

\subsection{parEvalNLazy}
\code{parEvalN} is 100\% strict, which means that it fully evaluates all passed arrows. Sometimes this might not be feasible, as it will not work on infinite lists of functions like e.g. \code{map (arr . (+)) [1..]} or just because we need the arrows evaluated in chunks. \code{parEvalNLazy} fixes this. It works by first chunking the input from \code{[a]} to \code{[[a]]} with the given \code{ChunkSize} in \code{(arr \$ chunksOf chunkSize)}. These chunks are then fed into a list \code{[arr [a] [b]]} of parallel arrows created by feeding chunks of the passed \code{ChunkSize} into the regular parEvalN by using \code{listApp}. The resulting \code{[[b]]} is lastly converted into \code{[b]} with \code{arr concat}.
\begin{lstlisting}[frame=htrbl]
parEvalNLazy :: (ArrowParallel arr a b conf, ArrowChoice arr, ArrowApply arr) =>
	conf -> ChunkSize -> [arr a b] -> (arr [a] [b])
parEvalNLazy conf chunkSize fs =
	arr (chunksOf chunkSize) >>>
	listApp fchunks >>>
	arr concat
	where
		fchunks = map (parEvalN conf) $ chunksOf chunkSize fs
\end{lstlisting}

\subsection{parEval2}
We have only talked about parallelizing the computation of arrows of the same type up until now. But sometimes we want to paralellize heterogenous types as well. For this, we introduce a helper combinator \code{arrMaybe} first, that converts an arrow \code{arr a b} to an arrow \code{arr (Maybe a) (Maybe b)}.
\begin{lstlisting}[frame=htrbl]
arrMaybe :: (ArrowApply arr) => (arr a b) -> arr (Maybe a) (Maybe b)
arrMaybe fn = (arr $ go) >>> app
	where 
		go Nothing = (arr $ \Nothing -> Nothing, Nothing)
		go (Just a) = ((arr $ \(Just x) -> (fn, x)) >>> app >>> arr Just, (Just a))
\end{lstlisting}
With this, we can now easily write \code{parEval2} which combines two arrows \code{arr a b} and \code{arr c d} into a new parallel arrow \code{arr (a, c) (b, d)}. We start by converting both arrows with \code{arrMaybe}, combining them with \code{***} into a new arrow \code{arr (Maybe a, Maybe c) (Maybe b, Maybe d)}. This is then replicated twice and fed into parEvalN to get a \code{arr [(Maybe a, Maybe c)] [(Maybe b, Maybe d)]}. We can then apply this arrow to the input \code{[(Just a, Nothing), (Nothing, Just c)]} and then extract the resulting values with \code{fromJust} and the \code{!!} operator on lists in the last step.
\begin{lstlisting}[frame=htrbl]
parEval2 :: (ArrowParallel arr a b conf,
	ArrowParallel arr (Maybe a, Maybe c) (Maybe b, Maybe d) conf,
	ArrowApply arr) =>
	conf -> arr a b -> arr c d -> (arr (a, c) (b, d))
parEval2 conf f g =
	(arr $ \(a, c) -> (f_g, [(Just a, Nothing), (Nothing, Just c)])) >>>
	app >>>
	(arr $ \comb -> (fromJust (fst (comb !! 0)), fromJust (snd (comb !! 1))))
where
	f_g = parEvalN conf $ replicate 2 $ arrMaybe f *** arrMaybe g
\end{lstlisting}

\subsection{parMap}
\code{parMap} is probably the most common skeleton for parallel programs. We can implement it with \code{ArrowParallel} by repeating an arrow \code{arr a b} and then passing it into \code{parEvalN} to get an arrow \code{arr [a] [b]}. 
Just like \code{parEvalN}, \code{parMap} is 100 \% strict.
\begin{lstlisting}[frame=htrbl]
parMap :: (ArrowParallel arr a b conf, ArrowApply arr) =>
	conf -> (arr a b) -> (arr [a] [b])
parMap conf f =
	(arr $ \as -> (f, as)) >>>
	(first $ arr repeat >>>
		arr (parEvalN conf)) >>>
	app
\end{lstlisting}

\subsection{parMapStream}
As \code{parMap} is 100\% strict it has the same restrictions as \code{parEvalN} compared to \code{parEvalNLazy}. So it makes sense to also have a \code{parMapStream} which behaves like \code{parMap}, but uses \code{parEvalNLazy} instead of \code{parEvalN}.
\begin{lstlisting}[frame=htrbl]
parMapStream :: (ArrowParallel arr a b conf, ArrowChoice arr, ArrowApply arr) =>
	conf -> ChunkSize -> arr a b -> arr [a] [b]
parMapStream conf chunkSize f =
	(arr $ \as -> (f, as)) >>>
	(first $ arr repeat >>>
		arr (parEvalNLazy conf chunkSize)) >>>
	app
\end{lstlisting}

\subsection{farm}
\code{parMap} spawns every single computation in a new thread (at least for the instances of \code{ArrowParallel} we gave in this paper). This can be quite wasteful and a \code{farm} that equally distributes the workload over \code{numCores} workers (if numCores is greater than actualProcessorCount, the fastest processor(s) to finish will get more tasks) seems useful.
\begin{lstlisting}[frame=htrbl]
farm :: (ArrowParallel arr a b conf, ArrowParallel arr [a] [b] conf,
	ArrowChoice arr, ArrowApply arr) =>
	conf -> NumCores -> arr a b -> arr [a] [b]
farm conf numCores f =
	(arr $ \as -> (f, as)) >>>
	(first $ arr mapArr >>> arr repeat >>>
		arr (parEvalN conf)) >>>
	(second $ arr (unshuffle numCores)) >>>
	app >>>
	arr shuffle
\end{lstlisting}
The definition of \code{unshuffle} is
\begin{lstlisting}[frame=htrbl]
unshuffle :: Int
	-> [a]
	-> [[a]]
unshuffle n xs = [takeEach n (drop i xs) | i <- [0..n-1]]
\end{lstlisting}
, while \code{shuffle} is defined as:
\begin{lstlisting}[frame=htrbl]
shuffle :: [[a]]
	-> [a]
shuffle = concat . transpose
\end{lstlisting}
(These were taken from Eden's source code. \cite{eden_skel_shuffle})

\subsection{farmChunk}
As \code{farm} is basically just \code{parMap} with a different work distribution, it is, again, 100\% strict. So we define \code{farmChunk} which uses \code{parEvalNLazy} instead of \code{parEvalN} like this:
\begin{lstlisting}[frame=htrbl]
farmChunk :: (ArrowParallel arr a b conf, ArrowParallel arr [a] [b] conf,
	ArrowChoice arr, ArrowApply arr) =>
	conf -> ChunkSize -> NumCores -> arr a b -> arr [a] [b]
farmChunk conf chunkSize numCores f =
	(arr $ \as -> (f, as)) >>>
	(first $ arr mapArr >>> arr repeat >>>
		arr (parEvalNLazy conf chunkSize)) >>>
	(second $ arr (unshuffle numCores)) >>>
	app >>>
	arr shuffle
\end{lstlisting}