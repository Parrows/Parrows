\section{Skeletons}
With the \code{ArrowParallel} typeclass in place and implemented, we can now implement some basic parallel skeletons.

\subsection{parEvalNLazy}
\begin{lstlisting}[frame=htrbl]
parEvalNLazy :: (ArrowParallel arr a b conf, ArrowChoice arr, ArrowApply arr) =>
	conf -> ChunkSize -> [arr a b] -> (arr [a] [b])
parEvalNLazy conf chunkSize fs =
	arr (chunksOf chunkSize) >>>
	listApp fchunks >>>
	arr concat
	where
		fchunks = map (parEvalN conf) $ chunksOf chunkSize fs
\end{lstlisting}
\frbreak

\subsection{parEval2}
\begin{lstlisting}[frame=htrbl]
arrMaybe :: (ArrowApply arr) => (arr a b) -> arr (Maybe a) (Maybe b)
arrMaybe fn = (arr $ go) >>> app
	where 
		go Nothing = (arr $ \Nothing -> Nothing, Nothing)
		go (Just a) = ((arr $ \(Just x) -> (fn, x)) >>> app >>> arr Just, (Just a))
\end{lstlisting}
\frbreak

\begin{lstlisting}[frame=htrbl]
parEval2 :: (ArrowParallel arr a b conf,
	ArrowParallel arr (Maybe a, Maybe c) (Maybe b, Maybe d) conf,
	ArrowApply arr) =>
	conf -> arr a b -> arr c d -> (arr (a, c) (b, d))
parEval2 conf f g =
	(arr $ \(a, c) -> (f_g, [(Just a, Nothing), (Nothing, Just c)])) >>>
	app >>>
	(arr $ \comb -> (fromJust (fst (comb !! 0)), fromJust (snd (comb !! 1))))
where
	f_g = parEvalN conf $ replicate 2 $ arrMaybe f *** arrMaybe g
\end{lstlisting}

\frbreak
\subsection{parMap}
\begin{lstlisting}[frame=htrbl]
parMap :: (ArrowParallel arr a b conf, ArrowApply arr) =>
	conf -> (arr a b) -> (arr [a] [b])
parMap conf f =
	(arr $ \as -> (f, as)) >>>
	(first $ arr repeat >>>
		arr (parEvalN conf)) >>>
	app
\end{lstlisting}

\frbreak

\subsection{parMapStream}
\begin{lstlisting}[frame=htrbl]
parMapStream :: (ArrowParallel arr a b conf, ArrowChoice arr, ArrowApply arr) =>
	conf -> ChunkSize -> arr a b -> arr [a] [b]
parMapStream conf chunkSize f =
	(arr $ \as -> (f, as)) >>>
	(first $ arr repeat >>>
		arr (parEvalNLazy conf chunkSize)) >>>
	app
\end{lstlisting}

\frbreak
\subsection{farm}
\begin{lstlisting}[frame=htrbl]
farm :: (ArrowParallel arr a b conf, ArrowParallel arr [a] [b] conf,
	ArrowChoice arr, ArrowApply arr) =>
	conf -> NumCores -> arr a b -> arr [a] [b]
farm conf numCores f =
	(arr $ \as -> (f, as)) >>>
	(first $ arr mapArr >>> arr repeat >>>
		arr (parEvalN conf)) >>>
	(second $ arr (unshuffle numCores)) >>>
	app >>>
	arr shuffle
\end{lstlisting}
The definition of \code{unshuffle} is
\begin{lstlisting}[frame=htrbl]
unshuffle :: Int
	-> [a]
	-> [[a]]
unshuffle n xs = [takeEach n (drop i xs) | i <- [0..n-1]]
\end{lstlisting}
, while \code{shuffle} is defined as:
\begin{lstlisting}[frame=htrbl]
shuffle :: [[a]]
	-> [a]
shuffle = concat . transpose
\end{lstlisting}
(These were taken from Eden's source code. \cite{eden_skel_shuffle})

\frbreak

\subsection{farmChunk}
\begin{lstlisting}[frame=htrbl]
farmChunk :: (ArrowParallel arr a b conf, ArrowParallel arr [a] [b] conf,
	ArrowChoice arr, ArrowApply arr) =>
	conf -> ChunkSize -> NumCores -> arr a b -> arr [a] [b]
farmChunk conf chunkSize numCores f =
	(arr $ \as -> (f, as)) >>>
	(first $ arr mapArr >>> arr repeat >>>
		arr (parEvalNLazy conf chunkSize)) >>>
	(second $ arr (unshuffle numCores)) >>>
	app >>>
	arr shuffle
\end{lstlisting}