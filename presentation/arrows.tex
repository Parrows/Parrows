\section{Arrows}
Another way to compose computations are arrows:
\subsection{Arrow Definition}
\begin{lstlisting}[frame=htrbl]
class Arrow arr where
	arr :: (a -> b) -> arr a b
	(>>>) :: arr a b -> arr b c -> arr a c
	first :: arr a b -> arr (a,c) (b,c)
\end{lstlisting}

Functions \lstinline{(->)} are arrows:
\begin{lstlisting}[frame=htrbl]
instance Arrow (->) where
arr f = f
f >>> g = g . f
first f = \(a, c) -> (f a, c)
\end{lstlisting}
\frbreak
The Kleisli type
\begin{lstlisting}[frame=htrbl]
data Kleisli m a b = Kleisli { run :: a -> m b }
\end{lstlisting}
is also an arrow:
\begin{lstlisting}[frame=htrbl]
instance Monad m => Arrow (Kleisli m) where
arr f = Kleisli $ return . f
f >>> g = Kleisli $ \a -> f a >>= g
first f = Kleisli $ \(a,c) -> f a >>= \b -> return (b,c)
\end{lstlisting}

\frbreak
\subsection{Combinators \& Arrow Example}
Some Combinators:
\begin{lstlisting}[frame=htrbl]
second :: Arrow arr => arr a b -> arr (c, a) (c, b)
second f = arr swap >>> first f >>> arr swap
where swap (x, y) = (y, x)
\end{lstlisting}
\begin{lstlisting}[frame=htrbl]
(***) :: Arrow arr => arr a b -> arr c d -> arr (a, c) (b, d)
f *** g = first f >>> second g
\end{lstlisting}
\begin{lstlisting}[frame=htrbl]
(&&&) :: Arrow arr => arr a b -> arr a c -> a a (b, c)
f &&& g = arr (\a -> (a, a)) >>> (f *** g)
\end{lstlisting}

Arrow usage example:
\begin{lstlisting}[frame=htrbl]
add :: Arrow arr => arr a Int -> arr a Int -> arr a Int
add f g = (f &&& g) >>> arr (\(u, v) -> u + v)
\end{lstlisting}
\frbreak